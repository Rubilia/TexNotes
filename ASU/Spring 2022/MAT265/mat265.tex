\documentclass[12pt]{article}
\usepackage{mathtools}
\usepackage[utf8]{inputenc}
\usepackage{amsfonts}
\usepackage[thinc]{esdiff}
\usepackage[makeroom]{cancel}
\usepackage{tikz}
\usepackage{pgfplots}


\DeclarePairedDelimiter\abs{\lvert}{\rvert}%
\newcommand*\circled[1]{\tikz[baseline=(char.base)]{
		\node[shape=circle,draw,inner sep=2pt] (char) {#1};}}
\newcommand{\bcancelto}[3][]{\tikz[baseline=(N.base)]{
		\node[main node](N){$#2$};
		\node[label node,#1, anchor=north west] at (N.south east){$#3$};
		\draw[strike out,-latex,#1]  (N.north west) -- (N.south east);
}}
\newcommand{\RNum}[1]{\uppercase\expandafter{\romannumeral #1\relax}}

\newenvironment{task}[1][0]{\vspace{.5cm} {\textbf{№ #1} \vspace{.5cm}\\ }\large}{}

\begin{document}
\part{Functions and limits}
\section{Continuity}
{\hfill \textit{12/01/2022}\vspace{.5cm}\\}
A function is continuous if $\lim\limits_{x \to a} f(x) = f(a)$\\
(It can be written as $\lim\limits_{x \to a} f(x) = f(\lim\limits_{x \to a} x)$)\\\\
\large{\textit{List of continuous functions:}}
\begin{itemize}
	\item polynomials (everywhere)
	\item rational functions
	\item exponential functions (everywhere)
	\item logarithms
	\item trigonometric functions ($sinx, \enspace cosx,$ etc)
	\item the inverse tangent functions
\end{itemize}
Any function that could be written as a combination of above continuous functions using arithmetic operations is continuous

\newpage
{\hfill \textit{14/01/2022}\vspace{.5cm}\\}

\begin{task}[1]
\begin{itemize}
	\item {$\lim\limits_{x \to 1} (4x^2-7x+5) = 2$	
}
\item{
$\lim\limits_{x \to 0} e^x(3+x^2+sinx) = 3$
}
\item{
$\lim\limits_{t \to 2} \sqrt{\frac{2t^2+1}{3t-2}}= \frac{3}{2}$
}
\end{itemize}
\end{task}


\section{Piecewise functions}
\begin{task}[2]
$f(x) = \left\{
\begin{array}{l}
	x^2+3, \quad x<-2\\
	5, \quad x=-2\\
	-3x+1,\quad -2 < x < 1\\
	4-x^2, \quad x\geq1\\
\end{array}\\
\right.\\
f(2)=4-2^2=0\\
a=-4, \quad a=-2, \quad a=1$  is f(x) continuous at x=1?\\
\begin{enumerate}
	\item {$a=-4\\
\left.
\begin{array}{l}
f(-4)=19\\
\lim\limits_{x \to -4^-} f(x)=\lim\limits_{x \to -4^-}(x^2+3)=19\\
\lim\limits_{x \to -4^+} f(x)=\lim\limits_{x \to -4^+}(x^2+3)=19\\
\end{array} \right\} \implies \lim\limits_{x \to -4} f(x)=19\\
$}
\item{$a=-2\\
\left.
\begin{array}{l}
f(-2)=5\\
\lim\limits_{x \to -2^-} f(x) = \lim\limits_{x \to -2^-} (x^2+3)=7\\
\lim\limits_{x \to -2^+} f(x) = \lim\limits_{x \to -2^+} (-3x+1)=7\\
\end{array}\right\} \implies \lim\limits_{x \to -2} f(x) = 7\\
$}
\item {$a=1\\
\left.
\begin{array}{l}
f(1)=3\\
\lim\limits_{x \to 1^-} f(x) = \lim\limits_{x \to 1^-} (1-3x)=-2\\
\lim\limits_{x \to 1^+} f(x) = \lim\limits_{x \to 1^+} (4-x^2)=3\\
\end{array}	\right\} \implies \lim\limits_{x \to 1}f(x)\text{  DNE}\\
$}
\end{enumerate}
\end{task}

\vspace{2cm}
{\hfill \textit{18/01/2022}\vspace{.5cm}\\}

\begin{task}[1]
Suppose that $g(x)= \left\{
\begin{array}{l}
sin(\pi x)+c\quad\text{ if } x \leq 2\\
x^2-c\quad \text{ if } x > 2\\
\end{array}
\right.$\\
For which value of c is g(x) a continuous function?\\
\vspace{.5cm}\\
g(x) is continuous at x=2 $\enspace \leftrightarrow\\ \vspace{.2cm} \leftrightarrow
\left\{ 
\begin{array}{l}
\lim\limits_{x \to 2} g(x)=g(2)=sin(\pi \cdot 2) + c = c\\
\lim\limits_{x \to 2^-} g(x) = \lim\limits_{x \to 2^-} (sin(\pi x) + c) = c\\
\lim\limits_{x \to 2^+} g(x) = \lim\limits_{x \to 2^+} (x^2 - c) = 4-c\\
\end{array}\right. \leftrightarrow 4-c=c \leftrightarrow \underline{c = 2}$
\end{task}

\begin{task}[2]
Prove that there`s a real number x in $[1, 3]$ such that\\ $x^2+sin(\pi x)=5$\\
(How would you approximate this number?)\\\\
Solution: Check two things:
\begin{enumerate}
	\item is $f(x)$ continuous at all points between a and b? YES
	\item is c between $f(a)$ and $f(b)$? YES - \\$f(1)=1<5<f(3)=9$
\end{enumerate}
By the \textit{Intermediate value theorem}, there is x $\in \enspace [1, 3]$ such that $f(x)=5$\\\\\\

How could I approximate a solution to f(x)=5?\\
Bisection method

\end{task}

\begin{task}[3]
$\lim\limits_{x \to 2} \frac{x^2+x-6}{x^2-3x+2} = \lim\limits_{x \to 2} \frac{\cancel{(x-2)}(x+3)}{\cancel{(x-2)}(x-1)}=-5$
\end{task}

\begin{task}[4]
$\lim\limits_{x \to 16} \frac{16-x}{\sqrt{x}-4}=\lim\limits_{x \to 16} -\frac{\cancel{(16-x)}(\sqrt{x}+4)}{\cancel{16-x}}=-8$
\end{task}

\newpage
{\hfill \textit{27/01/2022}\vspace{.5cm}}
\section{Derivatives}
\subsection{Exponential functions}
$\frac{d}{dx}\left[a^x\right] = \lim\limits_{h \to 0} \frac{a^{x+h}-a^x}{h}=\lim\limits_{h \to 0} \frac{a^h-1}{h} \cdot a^x = C(a) \cdot a^x, \quad$ C(a) - constant\\
$C(a) = 1$ when $h\approx(1+h)^{\frac{1}{h}} \leftrightarrow a=e$
\subsection{Trigonometric functions}
$\frac{d}{dx}\left[sin(x)\right] = \lim\limits_{h \to 0} \frac{sin(x+h)-sin(x)}{h}=\lim\limits_{h \to } \frac{sin(x)cos(h) + cos(x)sin(h) - sin(x)}{h} =\\= \lim\limits_{x \to 0} \left(cos(x) \cdot \frac{sin(h)}{h} +\\+ sinx() \cdot \frac{cos(h)-1}{h}\right) = cos(x) \cdot \lim\limits_{h \to 0} \frac{sinh}{h} + sin(x) \cdot \lim\limits_{h \to 0} \frac{cos(h) - 1}{h} = cos(x)\\
\frac{d}{dx}\left[cos(x)\right]=-sin(x)$

\subsection{Transformation rules}
\begin{itemize}
	\item {$\frac{d}{dx}\left[c \cdot f(x)\right] = c \cdot \frac{d}{dx}\left[f(x)\right]$}
	\item {$\frac{d}{dx} \left[f(x)+g(x)\right] = \frac{d}{dx} \left[f(x)\right] + \frac{d}{dx}\left[g(x)\right]$}
\end{itemize}

\vspace{1.5cm}
\begin{task}[1]
\begin{description}
	\item[a.] $\frac{d}{dx} \left[3x^2-2cos(x)\right]=6x+2sin(x)$
	\item[b.] $\frac{d}{dx} \left[\frac{(x-3)^2}{x}\right] = \frac{d}{dx} \left[x-6+\frac{9}{x}\right]=1-\frac{9}{x^2}$
	\item[c.] $\frac{d}{dx} \left[\frac{8e^{100}+\pi ^ {\pi} +7}{\sqrt{cos(\pi^8)-\sqrt{e}}}\right]=0$
\end{description}

\end{task}
\end{document}