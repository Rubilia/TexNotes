\documentclass[12pt]{article}
\usepackage{mathtools}
\usepackage[utf8]{inputenc}
\usepackage[russian]{babel}
\usepackage{amsfonts}
\usepackage[thinc]{esdiff}
\usepackage{tikz}
\usepackage{cancel}
\usepackage{amsmath}
\usepackage{xcolor}
\usepackage{pgfplots}
\pgfplotsset{compat=1.15}
\usepackage{mathrsfs}
\usetikzlibrary{arrows}


\DeclarePairedDelimiter\abs{\lvert}{\rvert}%
\newcommand{\RNum}[1]{\uppercase\expandafter{\romannumeral #1\relax}}
\newenvironment{task}[1][0]{\vspace{.5cm} {\textbf{№ #1} \vspace{.5cm}\\ }}{}

\begin{document}
	
\section{Комплексные числа и полиномы}
	{\centering {\LARGE{Классная работа}}\\ 
	\hfill \textit{17.09.2021}\\ \vspace{1cm}}

\begin{task}[137]
	{\large Вычислить $(1-\frac{\sqrt{3}-i}{2})^{24}$}\\
	z = {$(1-\frac{\sqrt{3}-i}{2})^{24}$ = $[1 - (\frac{\sqrt{3}-i}{2})]^{24}$ = $[cos(0)+isin(0)-cos(-\frac{\pi}{6})-isin(-\frac{\pi}{6})]^{24}$ = $[cos(0)-cos(-\frac{\pi}{6}) + i(sin(0) - sin(-\frac{\pi}{6}))]^{24}$ =\\= $[-2sin(-\frac{\pi}{12})sin(\frac{\pi}{12}) + 2isin(\frac{\pi}{12})cos(-\frac{\pi}{12})]^{24}$}\\
	z = $[-2sin(\frac{\pi}{12})(sin(-\frac{\pi}{12})-icos(-\frac{\pi}{12}))]^{24}$\\\\
	$sin(-\frac{\pi}{12})-icos(-\frac{\pi}{12})$ = $-cos(\frac{5\pi}{12})$ - $isin(\frac{5\pi}{12})$\\\\
	z = $[2sin(\frac{\pi}{12})(cos(\frac{5\pi}{12}) + isin(\frac{5\pi}{12}))]^{24}=2^{24}*sin(\frac{\pi}{12})^{24}(cos(10\pi)+isin(10\pi))$ = $2\sqrt{\frac{1-\frac{\sqrt{3}}{2}}{2}}^{24}$ = $(2-\sqrt{3})^{12}$
\end{task}

\vspace{1cm}
{
{\textbf{№ 554 b}\\ \vspace{.25cm}\\}
\textit{f(x) = $(x-2)^4+4(x-2)^3+6(x-2)^2+10(x-2)+20$}

}
\newpage
\subsection{НОД}
Даны полиномы $f(x)$, $g(x)$. Если $f(x)\mod{q(x)}$ = 0, то q(x) - делитель f(x)\\
НОД - наибольший общий делитель $f(x)$ и $g(x)$\\
$f(x)$ = $g(x)q_1(x)+r_1(x)$\\
$g(x)$ = $r_1(x)q_2(x)+r_2(x)$\\
$r_1(x)$ = $r_2(x)q_3(x)+r_3(x)$\\
\textbf{...}\\
$r_{k-1}(x)$ = $r_k(x)q_{k+1}(x)$\\
НОД$(f(x), g(x))$ = $r_k(x)$

\vspace{1cm}
{
	{\textbf{№ 577 a} \vspace{.25cm}\\}
	\textit{НОД($x^4+x^3-3x^2-4x-1$, $x^3+x^2-x-1$)}\vspace{.25cm}\\
	
	$x^4+x^3-3x^2-4x-1$ = $x(x^3+x^2-x-1)-2x^2-3x-1$\\
	$(x^3+x^2-x-1)$ = $(2x^2+3x+1)(\frac{1}{2}x-\frac{1}{4})$ - $\frac{3}{4}(x+1)$\\
	$(2x^2+3x+1)$ = $(x+1)(2x+1)$ => НОД($x^4+x^3-3x^2-4x-1$, $x^3+x^2-x-1$) = x + 1
	
}

\vspace{1cm}
{
	{\textbf{№ 577 c} \vspace{.25cm}\\}
	\textit{НОД($x^6+7x^4+8x^3+7x+7$, $3x^5-7x^3+3x^2-x$)}\vspace{.25cm}\\
	
	$x^6+7x^4+8x^3+7x+7$ = $\frac{x}{3}(3x^5-7x^3+3x^2-x)-\frac{14}{3}x^4+7x^3-\frac{14}{3}x+7$\\
	$3x^5-7x^3+3x^2-x$ = $(2x^4-3x^3+2x-3)(\frac{3}{2}x+\frac{9}{4})-\frac{x^3+1}{4}$\\
	$2x^4-3x^3+2x-3$ = $(x^3+1)(2x-3)$\\
	НОД($x^6+7x^4+8x^3+7x+7$, $3x^5-7x^3+3x^2-x$) = $x^3+1$\\
	
	
}

\vspace{1cm}
\emph{ДЗ: 549c; 577e,f;  578c,d; 593c; 583a}

\newpage
{\centering {\LARGE{Классная работа}}\\ 
\hfill \textit{20.09.2021}\\ \vspace{1cm}}

\vspace{.75cm}
{
	{\textbf{№ 578 a} \vspace{.25cm}\\}
\textit{Найдем $\delta$:\\}
$x^{4}+2x^{3}-x^{2}-4x-2$ = ($x^{4}+x^{3}-x^{2}-2x-2$) + $x^{3}-2x$\\
$x^{4}+x^{3}-x^{2}-2x-2$ = ($x+1$)($x^{3}-2x$) + $x^{2}-2$\\
$x^{3}-2x$ = $x$($x^{2}-2$)\\
$\delta$ = GCD<$x^{4}+2x^{3}-x^{2}-4x-2$, $x^{4}+x^{3}-x^{2}-2x-2$> = $x^{2}-2$\vspace{.2cm}\\
$r_2 = g-r_1q_2$ \quad | \quad $r_1=f-gq_1$ => $r_2=g-q_2(f-gq_1) = -q_2f + (1+q_1q_2)g$ \vspace{.1cm} \\
$x^2-2 = -(x+1)f_1 + (x+2)f_2$
}

\vspace{.75cm}
{
	{\textbf{№ 583 a} \vspace{.25cm}\\}
$x^{3}-6x^{2}+12x-8$ = ($x-4$)($x^{2}-2x+1$) + $3x-4$\\
$x^{2}-2x+1$ = ($\frac{1}{3}x-\frac{2}{9}$)($3x-4$) + $\frac{1}{9}$\\
$3x-4$ = ($3x-4$)($1$)\\
GCD<$x^{3}-6x^{2}+12x-8$, $x^{2}-2x+1$> = $1$\\\\
$1 = 9(x^2-2x+1) - (3x-2)(3x-4) = 9(x-1)^2 -\\- (3x-2)((x-2)^3 - (x-4)(x-1)^2) = 
(x-1)^2(9 + (3x-2)(x-4)) -\\- (x-2)^3(3x-2)$ = $(x-1)^2(3x^2-14x+17)-(x-2)^3(3x-2)$	


}
\newpage
\subsection{Формула Виета.\\Разложение дроби на простейшие}
{\hfill \textit{20.09.2021}\\ \vspace{.2cm}}
$f_n=x^n+a_1x^{n-1}+a_2x^{n-2}+..+a_n$\\
\begin{equation}
	\begin{cases}
		\alpha_1+\alpha_2+...+\alpha_n=-a_1\\
		\alpha_1\alpha_2 + \alpha_1\alpha_3 + ... + \alpha_{n-1}\alpha{n}=a_2\\
		\alpha_1\alpha_2\alpha_3 + ... + \alpha_{n-2}\alpha_{n-1}\alpha_n=-a_3\\
		{\hfil .... \hfill}\\
		\alpha_1\alpha_2\alpha_3...\alpha_n = (-1)^{n}a_n\\
	\end{cases}
\end{equation}
\vspace{.75cm}\\
{
	{\textbf{Пример}\\}
	5, 2 - корни 1 кратности, 3 - 2 кратности. Найти $f(x)=x^4+a_1x^3+a_2x^2+a_3x+a_4$
	\begin{enumerate}
		\item 5-2+3+3=$-a_1$ => $a_1=-9$
		\item $5(-2) + 5\cdot3 + 5\cdot3 - 2\cdot3 - 2\cdot3 + 3\cdot3 = 17 = a_2$
		\item $a_3 = 33$
		\item $a_4=-90$
	\end{enumerate}
	$f(x)=x^4-9x^3+17x^2+33x-90$\\
}
\rule{\textwidth}{0.4pt}
\vspace{1cm}\\
{
	\large{$\frac{P(x)}{Q(x)}, \quad degP < degQ$\\}
	$Q(x)=(x-\alpha_1)^{k_1}(x-\alpha_2)^{k_2}...(x-\alpha_s)^{k_s}(x^2+p_1x+q_1)^{l_1}
	(x^2+p_2x+q_2)^{l_2}...(x^2+p_mx+q_m)^{l_m}$\\
		$\frac{P(x)}{Q(x)} = \frac{A_{11}}{x-\alpha_1} + \frac{A_{12}}{(x-\alpha_1)^2} + ... +
		\frac{A_{1k_1}}{(x-\alpha_1)^{k_1}}+\frac{A_{21}}{x-\alpha_2} + \frac{A_{22}}{(x-\alpha_2)^2} + ... + 	\frac{A_{2k_2}}{(x-\alpha_2)^{k_2}} + ... + 
		\frac{A_{s_{k_s}}}{(x-\alpha_s)^{k_s}} + \frac{B_{11}x+C_{11}}{x^2+p_1x+q_1}
		 + \frac{B_{12}x+C_{12}}{(x^2+p_1x+q_1)^2} + ... +  \frac{B_{1l_1}x+C_{1l_1}}{(x^2+p_1x+q_1)^{l_1}} + ... +\frac{B_{ml_m}x+C_{ml_m}}{(x^2+p_mx+q_m)^{l_m}}$
}

\newpage
{
	{\textbf{Пример 1}\\}
	\vspace{.2cm}\\
	$\frac{1}{x^3-1} = \frac{1}{x-1} \cdot \frac{1}{x^2+x+1}$\\
	$\frac{1}{x^3-1} = \frac{A}{x-1} + \frac{Bx+C}{x^2+x+1} = \frac{(A+B)x^2+x(A-B+C)+(A-C)}{x^3-1} \leftrightarrow$\\
	\begin{equation}
		\leftrightarrow
		\begin{cases}
			A+B=0\\
			A-B+C=0\\
			A-C=1\\
		\end{cases}
	\leftrightarrow
			\begin{cases}
				A = \frac{1}{3}\\
				B =  -\frac{1}{3}\\
				C = -\frac{2}{3}\\
			\end{cases}
	\end{equation}
}
\vspace{.75cm}\\
{
	{\textbf{№ 625c}\vspace{.2cm} \\}
	$\frac{5x^2+6x-23}{(x-1)^3(x+1)^2(x-2)} = \frac{A}{x-1} + \frac{B}{(x-1)^2} + \frac{C}{(x-1)^3} + \frac{D}{x+1} + \frac{E}{(x+1)^2} + \frac{F}{x-2}$\vspace{.2cm}\\
	$A(x-1)^2(x+1)^2(x-2)+B(x-1)(x+1)^2(x-2)+C(x+1)^2(x-2)+\\+D(x-1)^3(x+1)(x-2)+E(x-1)^3(x-2)+F(x-1)^3(x+1)^2=5x^2+6x-23$\\
	\begin{equation}
		\begin{cases}
			x = 1: \quad \enspace -4C=-12\\
			x = -1: \quad 24E=-24\\
			x = 2: \quad \enspace 9F=9\\
		\end{cases}
	\leftrightarrow
	\begin{cases}
		C=3\\E=-1\\ F=1\\
	\end{cases}
	\end{equation}
	\begin{equation}
		\begin{cases}
			x = 0: \quad -2A+2B-2C+2D+2E-F=-23 \leftrightarrow -A+B+D=-7\\
			x^5:\qquad \enspace A+D+1=0\\
			x^4:\qquad \enspace -2A+B-4D=2
		\end{cases}
	\leftrightarrow\\
	\end{equation}
	\begin{equation}
		\leftrightarrow
		\begin{cases}
			A+D=-1\\
			-2A+B-4D=2\\
			-A+B+D=-7\\
		\end{cases}
		\leftrightarrow
		\begin{cases}
			A=1\\
			B=-4\\
			D=-2\\
		\end{cases}
	\end{equation}
	$\frac{5x^2+6x-23}{(x-1)^3(x+1)^2(x-2)} = \frac{1}{x-1} - \frac{4}{(x-1)^2} + \frac{3}{(x-1)^3} - \frac{2}{x+1} - \frac{1}{(x+1)^2} + \frac{1}{x-2}$
}

\newpage
\section{Матрицы}
{\hfill \textit{27.09.2021}\\ \vspace{.8cm}\\}
	$A_{m \times n}$ \qquad ${a_{ij}}$\\
	$A_{n \times n}$  - квадратная\\
	$E_{n \times n} = $
	$\begin{pmatrix}
		1 & 0 & 0 \\
		0 & 1 & 0 \\
		0 & 0 & 1
	\end{pmatrix}$  - единичная\\
	$0 = $
	$\begin{pmatrix}
		0 & 0 & 0 \\
		0 & 0 & 0 \\
		0 & 0 & 0
	\end{pmatrix}$ - нулевая\\
	$D_{n \times n} = $
	$\begin{pmatrix}
		* & 0 & 0 \\
		0 & * & 0 \\
		0 & 0 & *
	\end{pmatrix}$ - диагональная\\
	\subsection{Операции над матрицами}
	$C_{m \times n} = A_{m \times n} \pm B_{m \times n}$, \qquad $c_{ij}=a_{ij} \pm b_{ij}$\\
	$C_{m \times n}=A_{m \times p} \cdot B_{p \times n}$\\
	$C = \alpha \cdot A$, \qquad $c_{ij} = \alpha \cdot a_{ij}$\\
	$A_{m \times n}^T \to A_{n \times n}$, \qquad $a_{ij} \to a_{ji}$\\
	\subsection{Определитель матрицы}
	$\abs{A}$ или $det(A)$ \vspace{.2cm}\\
		$\begin{vmatrix}
		a_{11} & a_{12} \\
		a_{21} & a_{22} \\
	\end{vmatrix}$ = $a_{11} \cdot a_{22} - a_{21} \cdot a_{12}$ \vspace{.2cm}\\
	$\begin{vmatrix}
	a_{11} & a_{12} & a_{13} \\
	a_{21} & a_{22} & a_{23} \\
	a_{31} & a_{32} & a_{33}
\end{vmatrix}$ = $(a_{11} \cdot a_{22} \cdot a_{33} + a_{12} \cdot a_{23} \cdot a_{31} + a_{21} \cdot a_32 \cdot a_{13}) -\\- (a_{13} \cdot a_{22} \cdot a_{31} + a_{23} \cdot a_{32} \cdot a_{11} + a_{21} \cdot a_{12} \cdot a_{33})$\\

% Helper image
\definecolor{myblue}{rgb}{0,0,0.8}
\definecolor{myred}{rgb}{0.8,0,0}


\begin{tikzpicture}
	\node[fill=none,draw=none] (matrix)
	{
		\begin{minipage}{\textwidth}
			\[ \begin{pmatrix} 
				a_1 & b_1 & c_1 \\
				a_2 & b_2 & c_2 \\
				a_3 & b_3 & c_3
			\end{pmatrix}
			\begin{matrix}
				a_1 & b_1 \\
				a_2 & b_2 \\
				a_3 & b_3
			\end{matrix} \]
		\end{minipage}
	};
	\draw[myblue,fill=myblue,opacity=0.2] (-1.4,-0.4)--(0.25,0.6)--(0.4,0.4)--(-1.25,-0.6);
	\draw[myblue,fill=myblue,opacity=0.2] (-0.7,-0.4)--(0.9,0.6)--(1.05,0.4)--(-0.55,-0.6);
	\draw[myblue,fill=myblue,opacity=0.2] (0,-0.4)--(1.55,0.6)--(1.7,0.4)--(0.15,-0.6);
	\draw[myred,fill=myred,opacity=0.2] (-1.4,0.4)--(0.25,-0.6)--(0.4,-0.4)--(-1.25,0.6);
	\draw[myred,fill=myred,opacity=0.2] (-0.7,0.4)--(0.85,-0.6)--(1.0,-0.4)--(-0.55,0.6);
	\draw[myred,fill=myred,opacity=0.2] (0,0.4)--(1.55,-0.6)--(1.7,-0.4)--(0.15,0.6);
	\node[above of=matrix,xshift=-1.3cm,yshift=-0.2cm] {\textcolor{myred}{$+$}};
	\node[above of=matrix,xshift=-0.6cm,yshift=-0.2cm] {\textcolor{myred}{$+$}};
	\node[above of=matrix,xshift=-0cm,yshift=-0.2cm] {\textcolor{myred}{$+$}};
	\node[above of=matrix,xshift=0.4cm,yshift=-0.2cm] {\textcolor{myblue}{$-$}};
	\node[above of=matrix,xshift=1cm,yshift=-0.2cm] {\textcolor{myblue}{$-$}};
	\node[above of=matrix,xshift=1.6cm,yshift=-0.2cm] {\textcolor{myblue}{$-$}};
	
\end{tikzpicture}

\vspace{.5cm}
{
	{\textbf{№ 220c} \vspace{.5cm}\\}
	$\begin{pmatrix}
		3 & 1 & 1 \\
		2 & 1 & 2 \\
		1 & 2 & 3
	\end{pmatrix} \cdot $
	$\begin{pmatrix}
		1 & 1 & -1 \\
		2 & -1 & 1 \\
		1 & 0 & 1
	\end{pmatrix}$ = 
	$\begin{pmatrix}
		6 & 2 & -1 \\
		6 & 1 & 1 \\
		8 & -1 & 4
	\end{pmatrix}$
}

\vspace{.5cm}
{
	{\textbf{№ 223a} \vspace{.5cm}\\}
	$\begin{pmatrix}
		2 & 1 & 1 \\
		3 & 0 & 1
	\end{pmatrix} \cdot $
	$\begin{pmatrix}
		3 & 1 \\
		2 & 1 \\
		1 & 0
	\end{pmatrix}$ = 
	$\begin{pmatrix}
		9 & 3 \\
		10 & 3 \\
	\end{pmatrix}$
}

\vspace{.5cm}
{
	{\textbf{№ 220c} \vspace{.5cm}\\}
	$\begin{pmatrix}
		1 & 2 & 1 \\
		0 & 1 & 2 \\
		3 & 1 & 1
	\end{pmatrix} \cdot $
	$\begin{pmatrix}
		2 & 3 & 1 \\
		-1 & 1 & 0 \\
		1 & 2 & -1
	\end{pmatrix} \cdot $
	$\begin{pmatrix}
	1 & 2 & 1 \\
	0 & 1 & 2 \\
	3 & 1 & 1
	\end{pmatrix}$ = 
	$\begin{pmatrix}
		1 & 7 & 0 \\
		1 & 5 & -2 \\
		6 & 12 & 2
	\end{pmatrix} \cdot$
	$\begin{pmatrix}
	1 & 2 & 1 \\
	0 & 1 & 2 \\
	3 & 1 & 1
	\end{pmatrix} = \\ =$
	$\begin{pmatrix}
	1 & 9 & 15 \\
	-5 & 5 & 9 \\
	12 & 26 & 32
	\end{pmatrix}$
}

\vspace{.5cm}
{
	{\textbf{№ 231c} \vspace{.5cm}\\}
	$\begin{vmatrix}
		sin(\alpha) & cos(\alpha) \\
		-cos(\alpha) & sin(\alpha)
	\end{vmatrix} = sin^2(\alpha) + cos^2(\alpha) = 1$

}

\vspace{.5cm}
{
	{\textbf{№ 232a} \vspace{.5cm}\\}
	$\begin{vmatrix}
		1 & 1 & 1 \\
		-1 & 0 & 1 \\
		-1 & -1 & 0
	\end{vmatrix} = -1 + 1 - (-1)=1$
	
}

\newpage
{
	{\textbf{№ 232e} \vspace{.5cm}\\}
	$\begin{vmatrix}
		1 & i & 1+i \\
		-i & 1 & 0 \\
		1-i & 0 & 1
	\end{vmatrix} = 1-(1+i)(1-i) - (-i^2) = 1 - 2 - 1=-2$
	
}

\vspace{.5cm}
{
	{\textbf{№ 273} \vspace{.5cm}\\}
	$\begin{vmatrix}
		13547 & 13647 \\
		28423 & 28523 \\
	\end{vmatrix} = \cancel{13547 \cdot 28423} + 1354700 - \cancel{28423 \cdot 13547} - 2842300
	=\\= 1354700 - 2842300 = -1487600$
	
}

\vspace{.5cm}
{
	{\textbf{№ 284} \vspace{.5cm}\\}
	$\begin{vmatrix}
		x & y & x+y \\
		y & x+y & x \\
		x+y & x & y
	\end{vmatrix} = 3x^2y+3xy^2-(x+y)^3-x^3-y^3 = -2(x^3+y^3)$\vspace{.5cm} \\
}
ДЗ: 220df, 274, 221, 224, 232cdf
\vspace{1.4cm}\\
	\subsubsection{Свойства определителей}
	\begin{enumerate}
		\item $det(A \cdot B) = det(A) \cdot det(B) \qquad \implies \qquad det(A^n)=det(A)^n$
		\item $det(A^T) = det(A)$
		\item $if \enspace \exists \enspace 0$ - строка(столбец) $\implies \enspace det(A) = 0$
		\item $if \enspace \exists \enspace $ две пропорциональные строки(столбца) $\implies \enspace det(A) = 0$
		\item При перестановке строк знак определителя меняется
		\item $\alpha \cdot a_{i}$ (строка/столбец) $\quad \implies \quad det = \alpha \cdot det(A)$
		\item $if \enspace a_i=alpha \cdot a_j + \beta \cdot a_k \enspace \implies \enspace det(A) = 0$
		\item Определитель не изменится, если к $a_i$ прибавить $\alpha \cdot a_j$ \\
		\item $\begin{vmatrix}
		a_{11} & a_{12} & ... & a_{1n} \\
		... & ... & ... & ...\\
		a_{i1} & a_{i2} & ... & a_{in} \\
		b_1+c_1 & b_2+c_2 & ... & b_n + c_n\\
		... & ... & ... & ...\\
		a_{n1} & a_{n2} & ... & a_{nn} \\
	\end{vmatrix} = 
	\begin{vmatrix}
		a_{11} & a_{12} & ... & a_{1n} \\
		... & ... & ... & ...\\
		a_{i1} & a_{i2} & ... & a_{in} \\
		b_1 & b_2 & ... & b_n\\
		... & ... & ... & ...\\
		a_{n1} & a_{n2} & ... & a_{nn} \\
	\end{vmatrix} + 
	\begin{vmatrix}
		a_{11} & a_{12} & ... & a_{1n} \\
		... & ... & ... & ...\\
		a_{i1} & a_{i2} & ... & a_{in} \\
		c_1 & c_2 & ... & c_n\\
		... & ... & ... & ...\\
		a_{n1} & a_{n2} & ... & a_{nn} \\
	\end{vmatrix}$
	\end{enumerate}
\vspace{.5cm}

\subsubsection{Подсчет определителя}

$A = \begin{pmatrix}
	1 & 2 & 3\\
	4 & 5 & 6\\
	7 & 8 & 9
\end{pmatrix} \qquad$
$\Delta_{12, 12} = \begin{vmatrix}
	1 & 2\\4 & 5
\end{vmatrix} \qquad$
$\Delta_{23, 23} = \begin{vmatrix}
	5 & 6\\8 & 9
\end{vmatrix} \qquad ...$\\
$A_{ij} = (-1)^{i+j} \cdot M_{ij}$\\

\[detA = \sum_{j=1}^{n} a_{ij} \cdot A_{ij}\] \hfill\\
\subsubsection{Примеры}

\vspace{.5cm}{
	{\textbf{№ 273} \vspace{.5cm}\\}
	\large{\emph{\RNum{1} Способ}}\vspace{.5cm}\\
	$\begin{vmatrix}
		-4 & 1 & 2 & -2 &1 \\
		0 & 3 & 0 & 1 & -5 \\
		2 & -3 & 1 & -3 & 1 \\
		-1 & -1 & 3 & -1 & 0 \\
		0 & 4 & 0 & 2 & 5 \\
	\end{vmatrix} =
	\begin{vmatrix}
		-4 & 1 & 2 & -2 &1 \\
		0 & 3 & 0 & 1 & -5 \\
		2 & -3 & 1 & -3 & 1 \\
		-1 & -1 & 3 & -1 & 0 \\
		0 & 7 & 0 & 3 & 0 \\
	\end{vmatrix} = 7 \cdot (-1)^{5+2} \cdot
	\begin{vmatrix}
	-4 & 2 & -2 & 1 \\
	0 & 0 & 1 & -5 \\
	2 & 1 & -3 & 1 \\
	-1 & 3 & -1 & 0 \\
	\end{vmatrix} + 3\cdot (-1)^{5+4} \cdot 
	\begin{vmatrix}
	-4 & 1 & 2 & 1 \\
	0 & 3 & 0 & -5 \\
	2 & -3 & 1 & 1 \\
	-1 & -1 & 3 & 0 \\
	\end{vmatrix} = -7 \cdot
	\begin{vmatrix}
		-4 & 2 & -2 & 1 \\
		0 & 0 & 1 & -5 \\
		0 & 4 & -8 & 3 \\
		0 & 10 & -2 & -1 \\
	\end{vmatrix}-\\ -3 \cdot [3 \cdot (-1)^{2 + 2} \cdot
	\begin{vmatrix}
		-4 & 2  & 1 \\
		2 & 1 & 1 \\
		-1 & 3 & 0 \\
	\end{vmatrix} -5 \cdot (-1)^{2+4} \cdot 
	\begin{vmatrix}
		-4 & 1  & 2 \\
		2 & -3 & 1 \\
		-1 & -1 & 3 \\
	\end{vmatrix}] = 28 \cdot 
	\begin{vmatrix}
	0 & 1  & -5 \\
	4 & -8 & 3 \\
	10 & -2 & -1 \\
	\end{vmatrix} - 9\cdot
	\begin{vmatrix}
		-4 & 2  & 1 \\
		2 & 1 & 1 \\
		-1 & 3 & 0 \\
	\end{vmatrix} + 15 \cdot
	\begin{vmatrix}
	-4 & 1  & 2 \\
	2 & -3 & 1 \\
	-1 & -1 & 3 \\
	\end{vmatrix} =\\= 28 \cdot (30 + 40 -400 + 4) -9 \cdot (-2 + 6 + 1 + 12) + 15 \cdot (36 -1 -4 -6 +4 -6) = -9128 - 153 + 225=\textcolor{red}{-1069 \quad \text{- Ошибка в вычислениях!}}$\vspace{.5cm}\\
\large{\emph{\RNum{2} Способ}}
\vspace{.5cm}\\
	$\begin{vmatrix}
	-4 & 1 & 2 & -2 &1 \\
	0 & 3 & 0 & 1 & -5 \\
	2 & -3 & 1 & -3 & 1 \\
	-1 & -1 & 3 & -1 & 0 \\
	0 & 4 & 0 & 2 & 5 \\
\end{vmatrix} = 
	\begin{vmatrix}
		-4 & 2\\
		2 & 1
	\end{vmatrix} \cdot (-1)^{4+3+1+3} \cdot
	\begin{vmatrix}
		3 & 1 & -5\\
		-1 & -1 & 0\\
		4 & 2 & 5\\
	\end{vmatrix} + 
\begin{vmatrix}
	-4 & 2\\
	-1 & 3
\end{vmatrix} \cdot (-1)^{1+4+1+3} \cdot 
\begin{vmatrix}
	3 & 1 & -5\\
	-3 & -3 & 1\\
	4 & 2 & 5
\end{vmatrix} + 
\begin{vmatrix}
	2 & 1\\
	-1 & 3
\end{vmatrix} \cdot (-1)^{3+4+1+4} \cdot 
\begin{vmatrix}
	1 & -2 & 1\\
	3 & 1 & -5\\
	-1 & -1 & 0
\end{vmatrix} = -1069
$


\newpage
\subsection{Собственные числа матрицы}
{\hfill \textit{22.10.2021}\\}
\large{
$\Delta_n=\alpha \cdot \Delta_{n-1}+\beta \cdot \Delta_{n-2}$\\
$\Delta_2=\lambda^2-1 \qquad \Delta_1 = \lambda \qquad \Delta_0=1$\\
$G(z)=\frac{1+\lambda z - \lambda z}{1- \lambda z + z^2} = \frac{1}{1-\lambda z +z^2}$ - \emph{производящая функция}\\
$1-\lambda z +z^2=0 \implies z_{1, 2} = \frac{\lambda \pm \sqrt{\lambda^2-4}}{2}$\\
$\lambda + \sqrt{\lambda^2-4}=(cos(\varphi) + isin(\varphi)) \leftrightarrow \cancel{\lambda ^2} -4 = 4(cos(2 \varphi) +isin(2 \varphi)) -4 \lambda(cos(\varphi) + isin(\varphi)) + \cancel{\lambda^2}$\\
$cos^2(\varphi)-sin^2(\varphi)+2sin(\varphi)cos(\varphi)-\lambda cos(\varphi)-\lambda i sin(\varphi)=-1 \leftrightarrow \lambda = 2cos(\varphi)$\\

$G(z) = \frac{A}{z_1-z}+\frac{B}{z_2-z}=\frac{A(z_2-z)+B(z_1-z)}{(z_1-z)(z_2-z)}$\\
\begin{equation*}
	\begin{cases*}
		z=z_1: \quad A(z_2-z_1)=1\\
		z=z_2: \quad B(z_1-z_2)=1\\
	\end{cases*}
	\leftrightarrow
	\begin{cases*}
		A(-2isin(\varphi))=1\\
		B(2isin(\varphi))=1
	\end{cases*} \implies
	\begin{cases*}
		A=\frac{i}{2sin(\varphi)}\\
		B = -\frac{i}{2sin(\varphi)}\\
	\end{cases*}
\end{equation*}\\
$\frac{1}{1-\alpha \cdot z} = \sum^{+\infty} (\alpha z)^n \implies$\\
$\frac{A}{z_1}\cdot \frac{1}{1-\frac{1}{z_1}z} = \frac{A}{z_1} \cdot \sum^{+\infty} (\frac{1}{z_1})^n 
\cdot z^n =A \cdot \sum^{+\infty} \frac{1}{z_1^{n+1}} \cdot z^n$\\
$G(z) = \frac{\frac{A}{z_1}}{1-\frac{1}{z_1}\cdot z} + \frac{\frac{B}{z_2}}{1-\frac{1}{z_2}\cdot z}= \frac{i}{2sin(\varphi)}\cdot \sum^{+\infty}[\frac{1}{(cos(\varphi)+isin(\varphi))^{n+1}} - \frac{1}{(cos(\varphi)-isin(\varphi))^{n+1}}]\cdot z^n \implies$
$\Delta_n = \frac{i}{2sin(\varphi)}\cdot[(cos(\varphi)-isin(\varphi))^{n+1} - (cos(\varphi)+isin(\varphi))^{n+1}] = \frac{i}{2sin(\varphi)}\cdot(-2sin((n+1)\varphi) = \frac{\sin((n+1)\varphi)}{sin(\varphi)}$\\
$\Delta_n=\frac{sin((n+1)\varphi)}{sin(\varphi)} = 0 \leftrightarrow (n+1)\varphi = \pi \cdot k, \qquad k \in \mathbb{Z}$\\
$\lambda = 2cos(\frac{\pi k}{n+1}), \qquad k \in \mathbb{Z}$

}
\newpage}
\subsection{Системы линейных алгебрарических уравнений}
\begin{equation*}
	\begin{cases*}
		a_{11}x_1+a_{12}x_2+ ... + a_{1n}x_n=b_1\\
		a_{21}x_1+a_{22}x_2+ ... + a_{2n}x_n=b_2\\
		...\\
		a_{m1}x_1+a_{m2}x_2+ ... + a_{mn}x_n=b_m\\
	\end{cases*}
\end{equation*}
$A = $
\ensuremath{
	\begin{pmatrix}
		a_{11} & ...  & a_{1n}\\
		&...&\\
		a_{m1} & ... & a_{mn}\\
	\end{pmatrix}
}
$\qquad X = $
\ensuremath{
	\begin{pmatrix}
		x_1\\
		.\\
		.\\
		x_n
	\end{pmatrix}
}
$\qquad b =$
\ensuremath{
	\begin{pmatrix}
		b_1\\
		.\\
		.\\
		b_m
	\end{pmatrix}
} \vspace{.4cm}\\
$Ax=0 $ - \emph{однородная система}\\
$Ax=b $ - \emph{неоднородная система}\\
\subsection{Решение систем}

\begin{enumerate}
	\item {
	$Ax=b, \qquad A_{[n \times n]} \implies X = A^{-1}b$	
}
	\item { \textit{Крамер}\\
		$Ax=b,  \qquad A_{[n \times n]}$\\
		$\Delta=det(A)$\\
		$\Delta_{i}=\Delta$, где вместо $i$-го столбца - столбец $b$\\
		$x_i=\frac{\Delta_i}{\Delta}$
}
	\item {\textit{Гаусс}\\
	$Ax=b,  \qquad A_{[m \times n]}$\\
	$(A|b) \rightarrow$ Трапецевидный вид\\
	Если нулевой строке соответствует ненулевой элемент в строке $b$, то решений нет\\
	Переписать систему в обычном виде, выразить одну переменную через все остальные\\
	$X_{\text{неоднородная}}=X_{\text{частное}}+X_{\text{общее}}$
}
\end{enumerate}

\newpage
\subsection{Примеры}

\vspace{.5cm}
{
	{\textbf{№ 449b} \vspace{.5cm}\\}
	\begin{equation*}
		\begin{cases*}
			x_1+x_2+x_3+2x_4=0\\
			2x_1-x_2-2x_3+x_4=0\\
		\end{cases*}
	\end{equation*}
\ensuremath{
	\begin{pmatrix}
		1 & 1 & 1 & 2 & | & 0\\
		2 & -1 & -2 & 1 & | & 0
	\end{pmatrix} \rightarrow_{S_2-2S_1}
	\begin{pmatrix}
		1 & 1 & 1 & 2 & | & 0\\
		0 & -3 & -4 & -3 & | & 0\\
	\end{pmatrix}
}\vspace{.5cm}\\
Система разрешима\\
	\begin{equation*}
	\begin{cases*}
		x_1+x_2+x_3+2x_4=0\\
		3x_2+4x_3+3x_4=0\\
	\end{cases*}
\leftrightarrow
	\begin{cases*}
	x_1=-x_2-x_3-2x_4=\frac{1}{3}c_3-c_4\\
	x_2 = \frac{1}{3}(-4c_3-3c_4)\\
	x_3=c_3\\
	x_4=c_4\\
\end{cases*}
\end{equation*}\\
$X = $
\ensuremath{
	\begin{pmatrix}
		\frac{1}{3}\\
		-1\\
		1\\
		0
	\end{pmatrix} \cdot c_3 +
	\begin{pmatrix}
	-1\\
	-1\\
	0\\
	1
	\end{pmatrix} \cdot c_4 \text{\emph{ - общее решение в параметрическом виде}}
}
$	\begin{pmatrix}
	-1\\
	-1\\
	0\\
	1
\end{pmatrix}$ и 
$	\begin{pmatrix}
	\frac{1}{3}\\
	-1\\
	1\\
	0
\end{pmatrix}$ - \emph{фундаментальная система решений (только при параметрах)}\\
Вектор без параметра - частное решение\\
}




\newpage

\subsection{Собственный вектор матрицы}
{\hfill \textit{22.10.2021}\\}
$Ax=\lambda X$\\
$(A-\lambda E)X=0$

\begin{task}[1032 e]
	$
	\begin{pmatrix}
		5 & 6 & -3\\
		-1 & 0 & 1\\
		1 & 2 & 1\\
	\end{pmatrix}\\
	f(\lambda) = 
	\begin{vmatrix}
		5 - \lambda & 6 & -3\\
		-1 & -\lambda & 1\\
		1 & 2 & 1-\lambda\\
	\end{vmatrix} = \lambda(\lambda-1)(5-\lambda) + 6+6-(-3\lambda)-2(5-\lambda)-\\-6(\lambda-1)=
	-\lambda^3 +6\lambda^2-12\lambda+8=0\\
	\lambda_{1,2,3}=2\\
	$\\
	$\lambda = 2$:\\
	$
\begin{vmatrix}
	3 & 6 & -3\\
	-1 & -2 & 1\\
	1 & 2 & -1\\
\end{vmatrix} \cdot X = 0\\
$

\begin{equation*}
	\begin{pmatrix}
		1 & 2 & -1 & | & 0\\
		-1 & -2 & 1 & | & 0\\
		1 & 2 & -1 & | & 0\\
	\end{pmatrix} \rightarrow 
	\begin{pmatrix}
		1 & 2 & -1 & | & 0\\
		0 & 0 & 0 & | & 0\\
		0 & 0 & 0 & | & 0\\
	\end{pmatrix} \rightarrow
	\begin{cases}
		x_1+2x_2-x_3=0\\
		x_1=c_1\\
		x_2=c_2\\
	\end{cases} \leftrightarrow
	\begin{cases}
	x_1=-2c_2-c_3\\
	x_1=c_1\\
	x_2=c_2\\
\end{cases}
\end{equation*}\\
\ensuremath{
X=c_1\cdot
\begin{pmatrix}
	-2\\
	1\\
	0\\
\end{pmatrix} + c_3 \cdot
\begin{pmatrix}
	1\\
	0\\
	1\\
\end{pmatrix}\\
\begin{pmatrix}
	-2 & 1 & 0\\
\end{pmatrix}^T, \qquad
\begin{pmatrix}
	1 & 0 & 1
\end{pmatrix}^T - \text{ФСР, собственные векторы}
}
\end{task}

\begin{task}[1032 j]
$
	\begin{pmatrix}
		0 & 0 & 1\\
		0 & 1 & 0\\
		1 & 0 & 0\\
	\end{pmatrix}\\
f(\lambda) = 
\begin{vmatrix}
	-\lambda & 0 & 1\\
	0 & 1-\lambda & 0\\
	1 & 0 & -\lambda\\
\end{vmatrix} = -\lambda^3+\lambda^2+\lambda-1=0\\
\lambda = \pm1\\$
\begin{itemize}
	\item {
	$\lambda = 1:\\
	\begin{pmatrix}
		-1 & 0 & 1\\
		0 & 0 & 0\\
		1 & 0 & -1
	\end{pmatrix} \cdot X = 0\\
	X = c_2 \cdot
	\begin{pmatrix}
		0 \\ 1 \\ 0\\
	\end{pmatrix} + c_3 \cdot
	\begin{pmatrix}
		1\\0\\1\\
	\end{pmatrix}\\$
}
\item{$\lambda = -1:$
}
\end{itemize}

{
\begin{equation*}
	\begin{cases}
		x_1+x_3=0\\
		x_2=0\\
	\end{cases} \leftrightarrow
\begin{cases}
	x_1=-c_3\\
	x_2=0\\
	x_3=c_3\\
\end{cases}
\end{equation*}\\
$X = c_3 \cdot
\begin{pmatrix}
	-1\\0\\1
\end{pmatrix}$}\\
Собственные векторы:
$\lambda=1:$
\end{task}

\newpage
\section{Геометрия}
\subsection{Векторы}
Вектор: $\vec{AB}, \quad \abs{\vec{AB}} -$ длина вектора\\
$\vec{a}$, $\vec{b}$ - коллинеарны, если лежат на одной или параллельных прямых\\
$\vec{a}$, $\vec{b}$ - компланарны, если лежат в одной или параллельных плоскостях\\

\begin{task}[9]
\begin{tikzpicture}
	\draw (0,0) node[anchor=north]{$A$}
	-- (4,-1) node[anchor=north]{$C$}
	-- (3,2) node[anchor=south]{$B$}
	-- cycle;
	\draw[->, thick] (7/3, 1/3) node[anchor=north]{$M$}
	-- (4, -1);
	\draw[->, thick] (7/3, 1/3)
	-- (0, 0);
	\draw[->, thick] (7/3, 1/3)
	-- (3, 2);
\end{tikzpicture}\\
Найти точку M: $\vec{MA}+\vec{MB}+\vec{MC}=\vec{0}$\\
$\vec{MA}=\vec{AB}-\vec{MB}$\\
$\vec{MC}=\vec{CB}-\vec{MB}$\\
Тогда: $\vec{AB}+\vec{CB}-3\vec{MB}=\vec{0}$\\
$\vec{MB} = \frac{\vec{AB}+\vec{CB}}{3}$
\end{task}

\begin{task}[23]
\begin{tikzpicture}[scale=2]
	\draw (0, -1) node[anchor=north]{$A$}
	-- (0.8666,-0.5) node[anchor=west]{$F$}
	-- (0.8666,0.5) node[anchor=west]{$E$}
	-- (0,1) node[anchor=south]{$D$}
	-- (-0.8666,0.5) node[anchor=east]{$C$}
	-- (-0.8666,-0.5) node[anchor=east]{$B$}
	-- cycle;
	\draw[->, thick] (0, -1) -- (-0.8666,-0.5);
	\draw[->, thick] (0, -1) -- (-0.8666,0.5);
	\draw (-0.4333, -0.75) node[anchor=north]{$\vec{e_1}$};
	\draw (-0.4333, -0.25) node[anchor=east]{$\vec{e_2}$};
\end{tikzpicture}\\
$\vec{AB}=e_1=(1,0)$\\
$\vec{AC}=e_2=(0, 1)$\\
\ensuremath{
\vec{BC}=\vec{e_2}-\vec{e_1}=(-1, 1)\\
\vec{CD}=\vec{AD}-\vec{AC}=2\vec{BC}-\vec{AC}=(-2, 1)\\
\vec{DE}=-\vec{e_1}=(-1, 0)\\
\vec{EF}=-\vec{BC}=(1, -1)\\
\vec{FA}=-\vec{CD}\\
}
\end{task}


{\hfill \textit{01.11.2021}\\}

\begin{task}[29]
$
\vec{a}, \vec{b}, \vec{c} -\text{Произвольные три вектора}\\
\alpha, \beta, \gamma\\
\vec{v_1} = \alpha \vec{a} - \beta \vec{b}, \quad \vec{v_2} = \gamma \vec{b} -\alpha \vec{c}, \quad \vec{v_3} = \beta \vec{c}-\gamma \vec{a} - \text{Доказать компланарность}\\
\text{Компланарность} \leftrightarrow \text{Линейная зависимость}\\
\gamma (\alpha \vec{a} - \beta \vec{b}) + \beta (\gamma \vec{b} -\alpha \vec{c}) + \alpha (\beta \vec{c}-\gamma \vec{a})=\vec{0} \implies \vec{v_1}, \vec{v_2}, \vec{v_3} - \text{линейно зависимы} 
\implies \\ \implies \vec{v_1}, \vec{v_2}, \vec{v_3} - \text{Компланарны}
$
\end{task}

\begin{task}[30]
$
\vec{a} = (2, 5, 14), \quad \vec{b}=(14, 5, 2)\\
$
Проекция $\vec{a}$ на $Oxy \parallel \vec{b}$ -?\\\\
$
\vec{d} = \vec{a}+\lambda \vec{b}\\
d_z=0, \text{т. к. }d \subset Oxy \implies d_z=a_z+\lambda b_z=14+2\lambda=0 \implies \lambda=-7 \implies\\ \implies \vec{d} = (-96, -30, 0)\\
$
\end{task}

\newpage
\begin{task}[28]
Установить, в каком случае тройки векторов будут линейно зависимы. Если они линейно зависимы - представить третий вектор как комбинацию двух. \\
\begin{enumerate}
	\item {
$\vec{a}=(5, 2, 1), \quad \vec{b}=(-1, 4, 2), \quad \vec{c}=(-1, -1, 6)\\
\underbrace{
	\begin{pmatrix}
		5 & 2 & 1\\
		-1 & 4 & 2\\
		-1 & -1 & 6\\
\end{pmatrix}}_{A} \cdot
\begin{pmatrix}
	\alpha\\
	\beta\\
	\gamma\\
\end{pmatrix} = 0\\$
Найдем ранг матрицы A:\\
$\left(
\begin{array}{ccc|c}
	5 & 2 & 1 & 0\\
	-1 & 4 & 2 & 0\\
	-1 & -1 & 6 & 0\\
\end{array} \right)
\: \longrightarrow \:
\left(\begin{array}{ccc|c}
	5 & 2 & 1 & 0\\
	0 & \frac{22}{5} & \frac{11}{5} & 0\\
	0 & 0 & \frac{13}{2} & 0\\
\end{array}\right)
\implies \alpha = \beta = \gamma = 0 \implies\\ \implies \vec{a}, \: \vec{b}, \: \vec{c} - \text{линейно независимы}
$
}
\item{
$
\vec{a} = (6, 4, 2), \quad \vec{b}=(-9, 6, 3), \quad \vec{c}=(-3, 6, 3)\\
\left(
\begin{array}{ccc|c}
	6 & 4 & 2 & 0\\
	-9 & 6 & 3 & 0\\
	-3 & 6 & 3 & 0\\
\end{array}\right) \longrightarrow
\left(\begin{array}{ccc|c}
	6 & 4 & 2 & 0\\
	0 &12 & 6 & 0\\
	0 & 0 & 0 & 0\\
\end{array}\right)
\implies \vec{a}, \vec{b}, \vec{c} - \text{линейно зависимы}\\
\begin{cases}
	\alpha = \frac{1}{2}\\
	\beta = \frac{2}{3}\\
	\gamma = -1\\
\end{cases} - \text{частное решение} \implies \vec{c} = \frac{1}{2} \vec{a} +
\frac{2}{3} \vec{b}\\
$
}
\end{enumerate}
\end{task}

\newpage
\subsubsection{Произведение векторов}
\begin{enumerate}
	\item {Умножение на число\\
	$\vec{p} = \alpha \cdot \vec{a}, \qquad \alpha \in \mathbb{R}$
}
	\item{Скалярное произведение\\
$\vec{a}\cdot\vec{b}=\left( \vec{a}, \vec{b}\right) = \abs{\vec{a}} \cdot \abs{\vec{b}} \cdot cos(\varphi)$\\
\begin{tikzpicture}
	\draw[fill=green!30] (0,0) -- (90:.75cm) arc (90:27:.75cm);
	\draw[->,color=black] (0,0) -- node[right=2pt] {$\vec{a}$} (27:2.2cm);
	\draw[->,color=black] (0,0) -- node[near end, right=-3pt] {$\vec{b}$} (90:2cm);
	\draw(60:0.5cm) node {$\varphi$};
\end{tikzpicture}\\
В декартовой системе координат: $\vec{a}=(x_1, y_1, z_1), \: \vec{b}=(x_2, y_2, z_2) \implies \vec{a} \cdot \vec{b}=x_1x_2+y_1y_2+z_1z_2$\\
Свойства
\begin{itemize}
	\item $(\vec{a}, \vec{b})=(\vec{b}, \vec{a})$
	\item $(\vec{a}, \vec{b} + \vec{c})=(\vec{a}, \vec{b}) + (\vec{a}, \vec{c})$
	\item $(\vec{a}, \alpha \vec{b}) = \alpha (\vec{a}, \vec{b})$
\end{itemize}
}
\item{Векторное произведение\\
$\vec{c}=\vec{a} \times \vec{b}=\left[\vec{a}, \vec{b}\right]$\\
\begin{tikzpicture}[scale=0.9]
	\draw[-,fill=white!95!red](0,0)--(3,0)--(4,1)--(1,1)--cycle;
	\node at (2,0.5) {$|\textcolor{blue}{a}\times \textcolor{red}{b}|$};
	\draw[ultra thick,-latex,blue](0,0)--(3,0)node[midway,below]{$a$};
	\draw[ultra thick,-latex,red](0,0)--(1,1)node[midway,above]{$b$};
	\draw[ultra thick,-latex,blue!50!red](0,0)--(0,3)node[pos=0.7,right]{$a\times b$};
	\draw (0.6,0) arc [start angle=0,end angle=45,radius=0.6]
	node[pos=0.7,right]{$\varphi$};
\end{tikzpicture}
\begin{enumerate}
	\item[1.] $\abs{\vec{c}} = \abs{\vec{a}} \cdot \abs{\vec{b}} \cdot sin(\varphi)$
	\item[2.] $\vec{c} \perp \vec{a}, \: \vec{c} \perp \vec{c}$
	\item[3.] $\vec{a}, \vec{b}, \vec{c}$ - правая тройка
\end{enumerate}
В декартовой системе координат: 
$\vec{a} \times \vec{b} =
\begin{vmatrix}
	\vec{i} & \vec{j} & \vec{k}\\
	x_1 & y_1 & z_1\\
	x_2 & y_2 & z_2\\
\end{vmatrix}$
\newpage
Свойства
\begin{itemize}
	\item $[\vec{a}, \vec{b}]=-[\vec{b}, \vec{a}]$
	\item $[\vec{a}, \vec{b} + \vec{c}]=[\vec{a}, \vec{b}] + [\vec{a}, \vec{c}]$
	\item $[\vec{a}, \alpha \vec{b}] = \alpha [\vec{a}, \vec{b}]$
\end{itemize}
}
\item {Смешанное произведение\\
$(\vec{a} \times \vec{b}) \cdot \vec{c}=\vec{a} \times (\vec{b} \cdot \vec{c})=(\vec{a}, \vec{b}, \vec{c})$\\
В декартовой системе координат: 
$\begin{vmatrix}
	x_1 & y_1 & z_1\\
	x_2 & y_2 & z_2\\
	x_3 & y_3 & z_3\\
\end{vmatrix} = (\vec{a}, \vec{b}, \vec{c})$
}
\item{Двойное векторное произведение\\
$\vec{a} \times \vec{b} \times \vec{c} = \vec{b}(\vec{a}, \vec{c}) - \vec{c}(\vec{a}, \vec{c})$
}
\end{enumerate}

\begin{task}[131]
$\triangle ABC, $ длины сторон = 1\\
$(\vec{AB}, \vec{BC}) + (\vec{BC}, \vec{CS})+(\vec{CA}, \vec{AB}) = 1\cdot 1 \cdot \cos(120) + 1\cdot 1 \cdot \cos(120º)+\\+ 1\cdot 1 \cdot \cos(120º)=-\frac{3}{2}$\\

\end{task}

\newpage
\subsection{Прямые и плоскости}
{\hfill \textit{08.11.2021}\vspace{.5cm}}

\emph{Взаимное расположение плоскостей/прямых на плоскости}
\begin{enumerate}
	\item { $\Pi_1 \perp \Pi_2$ ($l_1 \perp l_2$ на плоскости) $\implies\\
\implies \vec{n_1} \perp \vec{n_2} \leftrightarrow \vec{n_1} \cdot \vec{n_2} = 0\\
A_1 A_2 + B_1 B_2 +C_1 C_2 = 0 \:\: (A_1 A_2 + B_1 B_2 = 0)$
}
\item{$\Pi_1 \parallel \Pi_2$ ($l_1 \parallel l_2$ на плоскости) \\
	$\frac{A_1}{A_2} = \frac{B_1}{B_2} = \frac{C_2}{C_2}, \quad (\frac{A_1}{A_2} = \frac{B_1}{B_2})$
}
\item{$\Pi_1 \cap \Pi_2\\
\vec{n_1} \neq \vec{n_2}, \quad \vec{n_1} \times \vec{n_2} \neq 0
$
}
\end{enumerate}
\emph{Взаимное расположение прямых в пространстве}
\begin{enumerate}
	\item{ $l_1, l_2$ в одной плоскости\\
()
		
}
	\item {$l_1 \perp l_2 \implies a_1 \perp a_2 \implies a_1 \cdot a_2 = 0$
}
	\item{$l_1 \parallel l_2 \implies \vec{a_1} = \lambda \vec{a_2} \implies \frac{\alpha_1}{\alpha_2}=\frac{\beta_1}{\beta_2}=\frac{\gamma_1}{\gamma_2}$
}
\item{$l_1, l_2$ совпадают\\
	$\vec{a_1} \times \vec{a_2}=0 \quad (\vec{r_2} - \vec{r_1}) \times \vec{a_1}=0$
}
\end{enumerate}

\begin{task}[367]
$\triangle ABC: A(-2, 3), B(4, 1), C(6, -5), \quad m_a-?\\
M_A=(\frac{4+6}{2}, \frac{1-5}{2})=(5, -2) \implies m_a = (M_A A)\\
m_a: \frac{x+2}{7}=-\frac{y-3}{5} \leftrightarrow 5x+10+7y-21=0 \leftrightarrow 5x+7y-11=0\\
$
\end{task}

\begin{task}[372]
$y=ax+b \implies
\begin{cases}
	-3=4a+b\\
	S=\frac{1}{2} \abs{b} \abs{\frac{b}{a}}=3
\end{cases} \leftrightarrow 
\begin{cases}
	b^2=6\abs{a}\\
	b = -3-4a
\end{cases} \leftrightarrow
\begin{cases}
	b=-3-4a\\
	16a^2+24a-6\abs{a}+9=0\\
\end{cases}\leftrightarrow\\ \leftrightarrow
\left[
\begin{array}{c}
	\begin{cases}
		a>=0\\
		b=-3-4a\\
		16a^2+18a+9=0\\
	\end{cases} \leftrightarrow \emptyset\\
	\begin{cases}
		a<0\\
		b=-3-4a\\
		16a^2+30a+9=0\\
	\end{cases}
\end{array}
\right. \leftrightarrow
\left[
\begin{array}{l}
	\begin{cases}
		a=-\frac{3}{2}\\
		b=3\\
	\end{cases}\\
	\begin{cases}
		a=-\frac{3}{8}\\
		b=-\frac{3}{2}
	\end{cases}
\end{array}
\right.
$\\
$l_1: \: y=-\frac{3}{2}x+3 \leftrightarrow 3x+2y-6=0\\
l_2: \: y=-\frac{3}{8}-\frac{3}{2} \leftrightarrow 3x+8y+12=0$\\
\end{task}

\begin{task}[377]
$2x-y=0\\
5x-y=0\\
3x-y=0 - медиана$
У треугольника одна из вершин находится в точке $(0, 0)$. Через эту же вершину проходит медиана. Найдем две остальные вершины:\\
$A(0, 0), B(b, 2b), C(c, 5c)$. Тогда уравнение 3 стороны: $\frac{x-b}{c-b}=\frac{y-2b}{5c-2b} \implies\\\implies \frac{3-b}{c-b}=\frac{9-2b}{5c-2b}\\
M(\frac{b+c}{2}, \frac{2b+5c}{2})$ лежит на медиане $\implies
\frac{2b+5c}{2} = \frac{3b+3c}{2}\\
\begin{cases}
	2b+5c=3b+3c\\
	\frac{3-b}{c-b}=\frac{9-2b}{5c-2b}\\
\end{cases} \leftrightarrow
\begin{cases}
	b=2c\\
	-\frac{3-2c}{c}=\frac{9-4c}{c}
\end{cases} \leftrightarrow
\begin{cases}
	b=4\\
	c=2\\
\end{cases}\\$
Т. о. $B(4, 8), C(2, 10), (BC): x+y-12=0\\$ 
\end{task}

\newpage
{\hfill \textit{12.11.2021}\\}
\subsection{Система координат}
O, $\vec{e_1}, \vec{e_2}, \ldots, \vec{e_n}$\\
$\vec{x}=\alpha_1\vec{e_1}+\alpha_2\vec{e_2}+\ldots+\alpha_n\vec{e_n}\\
(\alpha_1, \alpha_2, \ldots, \alpha_n) - $ координаты вектора\\
\subsubsection{Смена систем координат}
O, $\vec{e_1}, \vec{e_2}, \vec{e_3}$\\
O', $\vec{e_1}', \vec{e_2'}, \vec{e_3'}$\\
M(x, y, z) в O, $\vec{e_1}, \vec{e_2}, \vec{e_3}$\\
M(x', y', z') в O', $\vec{e_1'}, \vec{e_2}', \vec{e_3'}$\\
O'$(x_0, y_0, z_0)$ в O, $\vec{e_1}, \vec{e_2}, \vec{e_3}$\\
$
\begin{cases}
	\vec{e_1}'=\sum \alpha_{1j}\vec{e_j}\\
	\vec{e_2}'=\sum \alpha_{2j}\vec{e_j}\\
	\vec{e_3}'=\sum \alpha_{3j}\vec{e_j}\\
\end{cases} \implies
A =
\begin{pmatrix}
	\alpha_{11} & \alpha_{12} & \alpha_{13}\\
	\alpha_{21} & \alpha_{22} & \alpha_{23}\\
	\alpha_{31} & \alpha_{32} & \alpha_{33}\\
\end{pmatrix}
$ - матрица преобразования\\
$
\vec{OM}=x\vec{e_1}+y\vec{e_2}+z\vec{e_3}\\
\vec{OM}=\vec{OO'} + \vec{O'M}\\
\vec{OO'}=x_0\vec{e_1}+y_0\vec{e_2}+z_0\vec{e_3}\\
\vec{O'M}=x'\vec{e_1'}+y'\vec{e_2'}+z'\vec{e_3'}\\
(x-x_0-x'\alpha_{11}-y'\alpha_{21}-z'\alpha_{31})\vec{e_1}+(y-y_0-x'\alpha_{12}-y'\alpha_{22}-z'\alpha_{32})\vec{e_2}+\\+(z-z_0-x'\alpha_{13}-y'\alpha_{23}-z'\alpha_{33})\vec{e_3}=0\leftrightarrow\\\leftrightarrow
\begin{cases}
	x=x_0+x'\alpha_{11}+y'\alpha_{21}+z'\alpha_{31}\\
	y=y_0+x'\alpha_{12}+y'\alpha_{22}+z'\alpha_{32}\\
	z=z_0+x'\alpha_{13}+y'\alpha_{23}+z'\alpha_{33}
\end{cases}\\
\begin{pmatrix}
	x\\
	y\\
	z\\
\end{pmatrix}=
\begin{pmatrix}
	x_0\\
	y_0\\
	z_0\\
\end{pmatrix}
+A^T\cdot
\begin{pmatrix}
	x'\\
	y'\\
	z'\\
\end{pmatrix} \implies 
\begin{pmatrix}
	x'\\
	y'\\
	z'\\
\end{pmatrix}=
(A^T)^{-1}\cdot
\begin{pmatrix}
	x-x_0\\
	y-y_0\\
	z-z_0\\
\end{pmatrix}
$

\begin{task}[91 (13)]
(2, 4), (-3, 0), (2, 1)\\
$
\frac{x+3}{-5}=\frac{y}{4} \leftrightarrow 5y-4x-12=0\\
5(y-1)-4(x-2)=0 \leftrightarrow 5y-4x-12=0\text{ - сторона треугольника}\\
$
Аналогично:\\
$
x=2 \leftrightarrow x=-3\text{ - вторая сторона}\\
\frac{x+3}{5}=y \leftrightarrow 5y-x-3=0 \leftrightarrow 5(y-2)-(x-4)=0 \leftrightarrow5y-x-3=0\text{ - третья сторона треугольника}\\
$
$
\begin{cases}
	x=-3\\
	5y-x-3=0
\end{cases} \leftrightarrow
\begin{cases}
	x=-3\\
	y=3
\end{cases}
$
 - 1 вершина\\
 $
 \begin{cases}
 	x=-3\\
 	5y-4x-12=0
 \end{cases} \leftrightarrow
\begin{cases}
	x=-3\\
	y=-3\\
\end{cases}
 $ - 2 вершина\\
 $
 \begin{cases}
 	5y-4x-12=0\\
 	5y-x-3=0\\
 \end{cases} \leftrightarrow
\begin{cases}
	x=7\\
	y=5
\end{cases}
 $ - 3 вершина
\end{task}

\begin{task}[119]
$
(4, \frac{\pi}{9}), (1, \frac{5\pi}{18})\\
\varphi = \frac{5\pi}{18} - \frac{\pi}{9} = \frac{\pi}{6}
S = \frac{1}{2} \cdot 4 \cdot 1 \cdot \frac{1}{2} = 1\\
$
\end{task}

\newpage
\begin{task}[664]
	
\definecolor{ududff}{rgb}{0.30196078431372547,0.30196078431372547,1}
\definecolor{qqwuqq}{rgb}{0,0.39215686274509803,0}
\begin{tikzpicture}[line cap=round,line join=round,>=triangle 45,x=1cm,y=1cm, scale=.8]
	\clip(-2,-3) rectangle (12.9,6.54);
	\draw [shift={(0,0)},line width=2pt,color=qqwuqq,fill=qqwuqq,fill opacity=0.10000000149011612] (0,0) -- (-18.43494882292201:0.6) arc (-18.43494882292201:29.74488129694223:0.6) -- cycle;
	\draw [->,line width=2pt] (0,0) -- (6,-2);
	\draw [->,line width=2pt] (0,0) -- (7,4);
	\draw [->,line width=2pt] (0,0) -- (9.662301569804855,0.9567564016288782);
	\draw [->,line width=2pt] (0,0) -- (-0.6045625891934577,6.105489385452906);
	\begin{scriptsize}
		\draw[color=black] (3.36,-0.65) node {x};
		\draw[color=black] (3.44,2.45) node {y};
		\draw[color=qqwuqq] (1.16,0.3) node {$\omega$};
		\draw[color=black] (6.56,1.03) node {x'};
		\draw[color=black] (0.04,3.75) node {y'};
		\draw [fill=ududff] (6.28,2.12) circle (2.5pt);
		\draw[color=ududff] (6.44,2.55) node {$M$};
	\end{scriptsize}
\end{tikzpicture}	

$
\vec{e_1'}=\frac{\vec{e_1}+\vec{e_2}}{\sqrt{2+2cos(\omega)}}=\frac{\vec{e_1} + \vec{e_2}}{2cos(\frac{\omega}{2})}\\
\vec{e_2'}=\frac{\vec{e_1}-\vec{e_2}}{\sqrt{2-2cos(\omega)}}=\frac{\vec{e_1} - \vec{e_2}}{2sin(\frac{\omega}{2})}\\
A = \begin{pmatrix}
	\frac{1}{2cos(\frac{\omega}{2})} & \frac{1}{2cos(\frac{\omega}{2})}\\
	-\frac{1}{2sin(\frac{\omega}{2})} & \frac{1}{2sin(\frac{\omega}{2})}
\end{pmatrix}\\
\begin{pmatrix}
	x'\\
	y'\\
\end{pmatrix} = (A^T)^-1\cdot 
\begin{pmatrix}
	x\\
	y\\
\end{pmatrix}\\
(A^T)^{-1}=
\begin{pmatrix}
	\frac{1}{2cos(\frac{\omega}{2})} & -\frac{1}{2sin(\frac{\omega}{2})}\\
	\frac{1}{2cos(\frac{\omega}{2})} & \frac{1}{2sin(\frac{\omega}{2})}
\end{pmatrix}^{-1}\\
\left(
\begin{array}{cc|cc}
	\frac{1}{2cos(\frac{\omega}{2})} & -\frac{1}{2sin(\frac{\omega}{2})} & 1 & 0\\
	\frac{1}{2cos(\frac{\omega}{2})} & \frac{1}{2sin(\frac{\omega}{2})} & 0 & 1\\
\end{array}\right) = 
\left(
\begin{array}{cc|cc}
	\frac{1}{2cos(\frac{\omega}{2})} & -\frac{1}{2sin(\frac{\omega}{2})} & 1 & 0\\
	0 & \frac{1}{sin(\frac{\omega}{2})} & -1 & 1\\
\end{array}\right)=\\=
\left(\begin{array}{cc|cc}
	1 & 0 & cos(\frac{\omega}{2}) & cos(\frac{\omega}{2})\\
	0 & 1 & -sin(\frac{\omega}{2}) & sin(\frac{\omega}{2})\\
\end{array}\right) \implies A^{-1} = 
\begin{pmatrix}
	cos(\frac{\omega}{2}) & cos(\frac{\omega}{2})\\
	-sin(\frac{\omega}{2}) & sin(\frac{\omega}{2})\\
\end{pmatrix}\\
\begin{pmatrix}
	x'\\
	y'\\
\end{pmatrix}=
\begin{pmatrix}
	cos(\frac{\omega}{2}) & cos(\frac{\omega}{2})\\
	-sin(\frac{\omega}{2}) & sin(\frac{\omega}{2})\\
\end{pmatrix} \cdot \begin{pmatrix}
x\\y\\
\end{pmatrix} \leftrightarrow\\\leftrightarrow
\begin{cases}
	x'= (x+y)cos(\frac{\omega}{2})\\
	y'= (y-x)sin(\frac{\omega}{2})
\end{cases}
$
\end{task}

\newpage
{\hfill \textit{12.11.2021}\\}

\begin{task}[665]
\definecolor{qqwuqq}{rgb}{0,0.39215686274509803,0}
\begin{tikzpicture}[line cap=round,line join=round,>=triangle 45,x=1cm,y=1cm, scale=0.7]
	\clip(-1.84551096236595832,-3.2189645516016605) rectangle (10.685017416843117,6.643749743869959);
	\draw [shift={(0,0)},line width=2pt,color=qqwuqq,fill=qqwuqq,fill opacity=0.10000000149011612] (0,0) -- (3.654966237010108:1.2098080732190425) arc (-6.654966237010108:49.74488129694222:0.6098080732190425) -- cycle;
	\draw [->,line width=1pt] (0,0) -- (6,-2);
	\draw [->,line width=1pt] (0,0) -- (7,4);
	\draw [->,line width=1pt] (0,0) -- (9.662301569804855,0.9567564016288782);
	\draw [->,line width=1pt] (0,0) -- (-0.6045625891934577,6.105489385452906);
	\draw [->,line width=1.8pt] (0,0) -- (-0.16844645405527364,1.7011440264997186);
	\draw [->,line width=1.8pt] (0,0) -- (1.736,0.992);
	\draw [->,line width=1.8pt] (0,0) -- (2.378463778937395,0.23551432648843781);
	\draw [->,line width=1.8pt] (0,0) -- (2.16,-0.72);
	\begin{scriptsize}
		\draw[color=black] (3.294811829793051,-0.766946246841057) node {$x'$};
		\draw[color=black] (4.401293627484465,2.8393647974865157) node {$y$};
		\draw[color=black] (5.958564305716827,0.8859463151424138) node {$x$};
		\draw[color=black] (-0.24319786899801593,4.301013591967767) node {$x$};
		\draw[color=qqwuqq] (1.6604835735264023,0.5307793183525771) node {$\omega$};
		\draw[color=black] (-0.3251594836418244,0.7113052387132082) node {$\vec{e_1'}$};
		\draw[color=black] (0.7130209685130833,0.8313052387132082) node {$\vec{e_2}$};
		\draw[color=black] (1.300412540127044,-0.13900963048972626) node {$\vec{e_1}$};
		\draw[color=black] (0.9589058124445088,-0.72543951915852174) node {$\vec{e_2'}$};
	\end{scriptsize}
\end{tikzpicture}\\
$Oxy \: \omega \rightarrow Ox'y', \quad\\ \abs{\vec{e_1}}=\abs{\vec{e_2}}=\abs{\vec{e_1'}}=\abs{\vec{e_2'}}=1\\
\vec{a}=\alpha \vec{e_2}-\vec{e_1}, \quad \vec{a} \parallel \vec{e_1'} \implies \vec{a}\cdot \vec{e_1}=0 \implies \alpha = \frac{1}{cos(\omega)}\\
\vec{a}=-\vec{e_1}+\frac{1}{cos(\omega)} \vec{e_2}\\
\abs{\vec{a}}=\sqrt{1+\frac{1}{cos^2(\omega)}-2}=tg(\omega) \implies\\\implies
\vec{e_1}'=\frac{\vec{a}}{\abs{\vec{a}}}=-ctg(\omega) \vec{e_1} + sin(\omega) \vec{e_2}\\$

$\vec{b}=\beta \vec{e_1} - e_2, \quad \vec{b} \parallel \vec{e_2'}\\
\implies \vec{b}\cdot \vec{e_2}=0 \leftrightarrow \beta \cdot cos(\omega) -1=0 \leftrightarrow \beta = \frac{1}{cos(\omega)}\\
\abs{\vec{b}}=\sqrt{\frac{1}{cos^2(\omega)} + 1 - 2}=tg(\omega) \implies\\ \implies
\vec{e_2'}=\frac{\vec{b}}{\abs{\vec{b}}}=\frac{1}{sin(\omega)} \vec{e_1} - ctg(\omega) \vec{e_2}\\\\
A = \begin{pmatrix}
	-ctg(\omega) & \frac{1}{sin(\omega)}\\
	\frac{1}{sin(\omega)} & -ctg(\omega)\\
\end{pmatrix}=A^T\\
(A^T)^{-1}=A^{-1}=
\left(
\begin{array}{cc|cc}
		-ctg(\omega) & \frac{1}{sin(\omega)} & 1 & 0\\
	\frac{1}{sin(\omega)} & -ctg(\omega) & 0 & 1\\
\end{array}
\right)=\left(
\begin{array}{cc|cc}
	1 & 0 & ctg(\omega) & \frac{1}{sin(\omega)}\\
	0 & 1 & \frac{1}{sin(\omega)} & ctg(\omega)\\
\end{array}
\right)\\
\begin{pmatrix}
	x'\\
	y'\\
\end{pmatrix}=A^{-1} \cdot 
\begin{pmatrix}
	x\\
	y\\
\end{pmatrix}\\\\
\begin{cases}
	x=-ctg(\omega)x'+\frac{1}{sin(\omega)}y'\\
	y=\frac{1}{sin(\omega)}x'-ctg(\omega)y'\\
\end{cases}\\
\begin{cases}
	x'=ctg(\omega)x+\frac{1}{sin(\omega)}y\\
	y'= \frac{1}{sin(\omega)}x + ctg(\omega)y\\
\end{cases}$
\end{task}

\vspace{2cm}
\subsection{Прямые и плоскости в пространстве}
Плоскости:\\
$
(\vec{r}-\vec{r_0}, \vec{n})=0, \: \leftrightarrow \: (\vec{r}, \vec{n})+D=0\\
\vec{r}=\vec{r_0}+\vec{a}t, \quad \vec{r}\cdot \vec{n}'-p=0, \: \abs{\vec{n'}}=1\\
d=\frac{\abs{\vec{r_1} \cdot \vec{n}}}{\abs{\vec{n}}}\\
$
Прямые:\\
$
\vec{r}=\vec{r_0}+\vec{a}t\\
\frac{x-x_0}{\alpha}=\frac{y-y_0}{\beta}=\frac{z-z_0}{\gamma}\\
\begin{cases}
	\vec{r} \cdot \vec{n_1} + D_1 = 0\\
	\vec{r} \cdot \vec{n_2} + D_2 = 0\\
\end{cases}\\
$
Уравнение пучка плоскостей, проходящих через прямую: \\
$\alpha (\vec{r} \cdot \vec{n_1} + D_1) + \beta (\vec{r} \cdot \vec{n_2} + D_2)=0$\\

\subsubsection{Примеры}

\begin{task}[513]
$\alpha\text{ - искомая плоскость}\\
l_1: \:\:\begin{cases}
	x=2+3t\\
	y=-1+6t\\
	z=4t\\
\end{cases}\\
l_2: \:\: \begin{cases}
	x=-1+2t=0\\
	y=3t\\
	z=-t\\
\end{cases}\\
Ax+By+Cz+D=0\\
\vec{n}=(A, B, C)\\
\vec{a_1}=(3, 6, 4)$ - напр. $l_1$\\
$\vec{a_2}=(2, 3, -1)$ - напр. $l_2$\\
$
\begin{cases}
	\vec{n} \cdot \vec{a_1}=0\\
	\vec{n} \cdot \vec{a_2} = 0\\
\end{cases} \leftrightarrow
\begin{cases}
	3A+6B+4C=0\\
	2A+3B-C=0\\
	C=1\\
\end{cases} \leftrightarrow
\begin{cases}
	A=6\\
	B = -\frac{11}{3}\\
	C = 1\\
\end{cases}\\
(2, -1, 0) \in \alpha \implies \alpha: 6(x-2)-\frac{11}{3}(y+1)+z=0 \leftrightarrow\\
\leftrightarrow \alpha: 18x-11y+3z-47=0\\
$
\end{task}

\begin{task}[587]
$
\Pi_1: 2x+3y-4z+5=0\\
\Pi_2: 2x-z+3=0\\
\Pi_3: x+y-z=0\\
\phi - ?\\
2x+3y-4z+5+\alpha(2x-z+3) = 0\text{ - пучок плоскостей через $\Pi_1 \cap \Pi_2$}\\
x(2+2\alpha) + 3y+z(-4-\alpha)z+5+3\alpha = 0 \implies \vec{n_{\phi}}=(2+2\alpha, 3, -4-\alpha)\\
\vec{a}=\vec{\phi \cap \Pi_3} = \vec{n_{\phi}} \times \vec{\Pi_3} = 
\begin{vmatrix}
	\vec{i} & \vec{j}, & \vec{k}\\
	2+2\alpha & 3 & -4-\alpha\\
	1 & 1 & -1\\
\end{vmatrix} = (1+\alpha, \alpha-2, 2\alpha-1)\\
\vec{b} = (\vec{\Pi_1 \cap \Pi_2}) = 
\begin{vmatrix}
	\vec{i} & \vec{j} & \vec{k}\\
	2 & 3 & -4\\
	2 & 0 & -1\\
\end{vmatrix} = (-3, -6, -6) \parallel (1, 2, 2)\\
\vec{a} \cdot \vec{b} = 1+\alpha +2\alpha - 4 + 4\alpha - 2=7\alpha-5=0 \implies \alpha = \frac{5}{7}\\
\phi: \frac{24}{7}+3y-\frac{33}{7}z+\frac{50}{7}=0 \leftrightarrow 24x+21y-33z+50=0\\
$
\end{task}
\end{document}
