\documentclass[a4paper, 12pt]{article}
\usepackage[T2A]{fontenc}% Внутренняя T2A кодировка TeX
\usepackage[utf8]{inputenc}% кодировка файла
\usepackage[russian]{babel}% поддержка переносов в русском языке
\usepackage{comment}% enables the use of multi-line comments (\ifx \fi) 
\usepackage{fullpage}% changes the margin
\usepackage[a4paper, total={7in, 10in}]{geometry}
\usepackage[fleqn]{amsmath}
\usepackage{amssymb,amsthm}  
\usepackage{graphicx}
\usepackage{verbatim}
\usepackage{float}
\usepackage{xcolor}
\usepackage{mdframed}
\usepackage[shortlabels]{enumitem}
\usepackage{indentfirst}
\usepackage{hyperref}
\usepackage{cancel}

\renewcommand{\thesubsection}{\thesection.\alph{subsection}}

\newenvironment{problem}[2][Задача]
{ \begin{mdframed}[backgroundcolor=gray!10] \textbf{#1 #2.} \\}
	{  \end{mdframed}}

\newenvironment{solution}
{\textit{Решение:}\vspace{.2cm}\\}
{\vspace{.1cm}\noindent\rule{7in}{1.5pt}}

\renewcommand{\qed}{\quad\qedsymbol}

\begin{document}
\noindent
\large\textbf{Домашняя работа №8} \hfill  Дата: 28.10.2021  \\
\noindent\rule{7in}{2pt}

%%%%%%%%%%%%%%%%%%%%%%%%%%%%%%%%%%%%%%%%%%%%%%%%%%%%
% Задача 1
\begin{problem}{293}
Вычислить определитель порядка $2n$
$\left| \begin{array}{rrrrrrr}a & 0 & 0 & \ldots & 0 & 0 & b\\ 0 & a & 0 & \ldots & 0 & b & 0 \\ 0 & 0 & a & \ldots & b & 0 & 0 \\ \vdots & \vdots & \vdots & \ddots & \vdots & \vdots & \vdots \\ 0 & 0 & b & \ldots & a & 0 & 0 \\ 0 & b & 0 & \ldots & 0 & a & 0 \\ b & 0 & 0 & \ldots & 0 & 0 & a \end{array} \right|$\\

\end{problem}
\begin{solution}
%%%%%%%%%%%%%%%%%%%%%%%%%%%%%%%%%%%%%%%%%%%%%%%%%%%%
%% Ваше решение задачи здесь

Докажем с помощью метода матиматической индукции, что\\ (1){$\Delta_{2n} = (a-b)\cdot(a+b)^{2n-1}$}\\
\emph{База:} $\Delta_2 = 
\begin{vmatrix}
	a & b\\
	b & a\\
\end{vmatrix} = a^2-b^2 = (a-b) \cdot (a+b)^{2 - 1}\\
$
\emph{Индукционный переход:} Пусть для некоторого $k=2n$ верно утверждение (1). Докажем его для $k=2n+2:$\vspace{.2cm}\\
$\Delta_k = a \cdot
\underbrace{
	\begin{vmatrix}
		a & 0 & 0 & \ldots & 0 & 0 & b\\ 
		0 & a & 0 & \ldots & 0 & b & 0 \\ 
		0 & 0 & a & \ldots & b & 0 & 0 \\ 
		\vdots & \vdots & \vdots & \ddots & \vdots & \vdots & \vdots \\ 
		0 & 0 & b & \ldots & a & 0 & 0 \\ 
		0 & b & 0 & \ldots & 0 & a & 0 \\ 
		b & 0 & 0 & \ldots & 0 & 0 & a\\
	\end{vmatrix}
}_{(k-1) \times (k-1)} + b \cdot (-1)^{(2n)} \cdot
\underbrace{
	\begin{vmatrix}
		a & 0 & 0 & \ldots & 0 & 0 & b\\ 
		0 & a & 0 & \ldots & 0 & b & 0 \\ 
		0 & 0 & a & \ldots & b & 0 & 0 \\ 
		\vdots & \vdots & \vdots & \ddots & \vdots & \vdots & \vdots \\ 
		0 & 0 & b & \ldots & a & 0 & 0 \\ 
		0 & b & 0 & \ldots & 0 & a & 0 \\ 
		b & 0 & 0 & \ldots & 0 & 0 & a\\
\end{vmatrix}}_{(k-1) \times (k-1)} = \Delta_{k-1} \cdot (a+b)\vspace{.2cm}\\$

Тогда: $\Delta_{2n+2} = (a+b) \cdot \Delta_{2n+1} = (a+b)^2 \cdot \Delta_{2n} 
= (a+b)^2 \cdot (a+b)^{2n-1} \cdot (a-b) =\\= (a+b)^{2(n+1)-1} \cdot (a-b)$\\

Таким образом $\Delta_{2n}= (a-b)\cdot(a+b)^{2n-1}$
\end{solution} 

%%%%%%%%%%%%%%%%%%%%%%%%%%%%%%%%%%%%%%%%%%%%%%%%%%%%
% Задача 2
\begin{problem}{296}
Вычислить определитель 
$\left| \begin{array}{cccccc}1 & 2 & 3 & \ldots & n-1 & n \\ 1 & 1 & 1 & \ldots & 1 & 1-n\\ 1 & 1 & 1 & \ldots & 1-n & 1 \\ \vdots & \vdots & \vdots & \ddots & \vdots & \vdots \\ 1 & 1-n & 1 & \ldots & 1 & 1 \end{array} \right|$

\end{problem}
\begin{solution}
%%%%%%%%%%%%%%%%%%%%%%%%%%%%%%%%%%%%%%%%%%%%%%%%%%%%
%% Ваше решение задачи здесь
$\begin{vmatrix}
	1 & 2 & 3 & \ldots & n-1 & n \\ 
	1 & 1 & 1 & \ldots & 1 & 1-n\\ 
	1 & 1 & 1 & \ldots & 1-n & 1 \\ 
	\vdots & \vdots & \vdots & \ddots & \vdots & \vdots\\ 
	1 & 1-n & 1 & \ldots & 1 & 1\\
\end{vmatrix} \:
\underset{\begin{subarray}{c}\text{$V_n-V_1$}\\
	\text{\ldots}\\\text{$V_2-V_1$}\end{subarray}}{=} \:
\underbrace{
\begin{vmatrix}
	1 & 1 & 2 & \ldots & n-2 & n-1\\
	1 & 0 & 0 & \ldots & 0 & -n\\
	1 & 0 & 0 & \ldots & -n & 0\\
	\vdots & \vdots & \vdots & \ddots & \ldots & \ldots\\
	1 & -n & 0 & \ldots & 0 & 0\\
\end{vmatrix}
}_{\large{\Delta_n}}\\
$
Чтобы не запутаться, заменим $-n$ на $-a$:\\
$
\Delta_n = 
\begin{vmatrix}
	1 & 1 & 2 & \ldots & n-2 & n-1\\
	1 & 0 & 0 & \ldots & 0 & -a\\
	1 & 0 & 0 & \ldots & -a & 0\\
	\vdots & \vdots & \vdots & \ddots & \ldots & \ldots\\
	1 & -a & 0 & \ldots & 0 & 0\\
\end{vmatrix} = (-1)^{n-1} \cdot (n-1) \cdot
{\underbrace{
\begin{vmatrix}
	1 & 0 & 0 & \ldots & 0\\
	1 & 0 & 0 & \ldots & -a\\
	\vdots & \vdots & \vdots & \ddots & \vdots\\
	1 & -a & 0 & \ldots & 0\\
\end{vmatrix}}_{\alpha_{n-1}}}_{(n-1) \times (n-1)} + \\ +
(-1)^{n+1} \cdot a \cdot 
\underbrace{
	\begin{vmatrix}
		1 & 1 & 2 & \ldots & n-2\\
		1 & 0 & 0 & \ldots & -a\\
		\vdots & \vdots & \vdots & \ddots & \vdots\\
		1 & 0 & -a & \ldots & 0\\
		1 & -a & 0 & \ldots & 0
	\end{vmatrix}
}_{\Delta_{n-1}}\\
\alpha_n = 
\begin{vmatrix}
	1 & 0 & 0 & \ldots & 0\\
	1 & 0 & 0 & \ldots & -a\\
	\vdots & \vdots & \vdots & \ddots & \vdots\\
	1 & -a & 0 & \ldots & 0\\
\end{vmatrix}_{n \times n} = 
\underbrace{
	\begin{vmatrix}
		0 & 0 & \ldots & -a\\
		0 & 0 & \ldots & 0\\
		\vdots & \vdots & \ddots & \vdots\\
		0 & -a & \ldots & 0\\
		-a & 0 & \ldots & 0
	\end{vmatrix}
}_{(-a)^{n-1}} + \\ +
\underbrace{
	\begin{vmatrix}
		0 & 0 & \ldots & 0\\
		0 & 0 & \ldots & 0\\
		\vdots & \vdots & \ddots & \vdots\\
		0 & -a & \ldots & 0\\
		-a & 0 & \ldots & 0
	\end{vmatrix} + \ldots +
	\begin{vmatrix}
		0 & 0 & \ldots & -a\\
		0 & 0 & \ldots & 0\\
		\vdots & \vdots & \ddots & \vdots\\
		0 & -a & \ldots & 0\\
		0 & 0 & \ldots & 0
	\end{vmatrix}
}_{0} = (-a)^{n-1} \quad \implies
$
\newpage
$
\implies \: \Delta_n = (-1)^{n+1} \cdot a \cdot \Delta_{n-1} + (-1)^{n-1} \cdot (-1)^{n-2} \cdot (n-1) \cdot a^{n-2} = (-1)^{n+1} \cdot a \cdot \Delta_{n-1} -\\-(n-1) \cdot a^{n-2} = 
(-1)^{n+1} \cdot a \cdot ((-1)^{n} \cdot a \cdot \Delta_{n-2} - (n-2) \cdot a^{n-3}) - (n-1) \cdot a^{n-2} =\\= -a^2 \cdot \Delta_{n-2} -a^{n-2} \cdot (n-1 + (-1)^{n+1} \cdot (n-2))
$
\vspace{.1cm}
\begin{enumerate}
	\item { $n \mod 4 = 0 \implies n=4k, \quad k \in \mathbb{N}$\\
$
\Delta_{4k} = -a^2 \cdot \Delta_{4k-2} - a^{4k-2} = -a^2 \cdot (-a^2 \cdot \Delta_{4k-4} - a^{4k-4}) - a^{4k-2} = a^4 \cdot \Delta_{4k-4}\\
\Delta_{4k} = a^4 \cdot \Delta_{4(k-1)} = a^8 \cdot \Delta_{4(k-2)} = \ldots = a^{k-1} \cdot \Delta_4\\
\Delta_3 = 
\begin{vmatrix}
	1 & 1 & 2\\
	1 & 0 & -a\\
	1 & -a & 0\\
\end{vmatrix} = 
\begin{vmatrix}
	0 & -a\\
	-a & 0\\
\end{vmatrix} - 
\begin{vmatrix}
	1 & 2\\
	-a & 0\\
\end{vmatrix} +
\begin{vmatrix}
	1 & 2\\
	0 & -a\\
\end{vmatrix} = -a^2 + 2a -a = -a^2+a\\
\Delta_4 = (-1)^5 \cdot a \cdot (a-a^2) - 3 a^{2} = a^3 - 4a^2\\
\Delta_{4k} = (a)^{k-1} \cdot (a^3-4a^2)=a^{k+1} \cdot (a-4) = n^{\frac{n}{4}+1} \cdot (n-4)
$
}
	\item { $n \mod 2 = 1$
}
\end{enumerate}

\end{solution} 

%%%%%%%%%%%%%%%%%%%%%%%%%%%%%%%%%%%%%%%%%%%%%%%%%%%%
% Задача 3
\begin{problem}{297}
Вычислить определитель 
$\left| \begin{array}{ccccc}1 & 2 & 3 & \ldots & n \\ 2 & 3 & 4 & \ldots & 1 \\ 3 & 4 & 5 & \ldots & 2 \\ \vdots & \vdots & \vdots & \ddots & \vdots \\ n & 1 & 2 & \ldots & n-1 \end{array} \right|$

\end{problem}
\begin{solution}
%%%%%%%%%%%%%%%%%%%%%%%%%%%%%%%%%%%%%%%%%%%%%%%%%%%%
%% Ваше решение задачи здесь


\end{solution} 

%%%%%%%%%%%%%%%%%%%%%%%%%%%%%%%%%%%%%%%%%%%%%%%%%%%%
% Задача 4
\begin{problem}{374(b)}
Вычислить определитель $\Delta$ посредством умножения на определитель $\delta$\\
$\Delta=\left| \begin{array}{rrrr} -1 & -9 & -2 & 3\\ -5 & 5 & 3 & -2\\ -12 & -6 & 1 & 1 \\ 9 & 0 & -2 & 1\end{array}\right|$;
$\delta=\left| \begin{array}{rrrr} 1 & 0 & 0 & 0\\ -2 & 1 & 0 & 0\\ 3 & 2 & 1 & 0 \\ -3 & 4 & 2 & 1\end{array}\right|$

\end{problem}
\begin{solution}
%%%%%%%%%%%%%%%%%%%%%%%%%%%%%%%%%%%%%%%%%%%%%%%%%%%%
%% Ваше решение задачи здесь


\end{solution} 

%%%%%%%%%%%%%%%%%%%%%%%%%%%%%%%%%%%%%%%%%%%%%%%%%%%%
% Задача 5
\begin{problem}{391}
Доказать, что $\det\left(\begin{array}{cc}E_m & B\\C & D\end{array}\right)=\det(D-CB)$.
Здесь $B$ и $C$~-- произвольные $m\times n$- и $n\times m$-матрицы, $D$~-- квадратная матрица порядка $n$.

\end{problem}
\begin{solution}
%%%%%%%%%%%%%%%%%%%%%%%%%%%%%%%%%%%%%%%%%%%%%%%%%%%%
%% Ваше решение задачи здесь


\end{solution} 

\end{document}