\noindent
\large\textbf{Домашняя работа №6} \hfill  Дата: 23.10.2021  \\
\noindent\rule{7in}{2pt}

%%%%%%%%%%%%%%%%%%%%%%%%%%%%%%%%%%%%%%%%%%%%%%%%%%%%
% Задача 1
\begin{problem}{220(d)}
 
Умножить матрицы: 
d)
$\left(
\begin{array}{rrr}
1 & 2 & 3 \\
2 & 4 & 6 \\
3 & 6 & 9 \\
\end{array}
\right)
\cdot
\left(
\begin{array}{rrr}
-1 & -2 & -4 \\
-1 & -2 & -4 \\
1 & 2 & 4 \\
\end{array}
\right)
$
\end{problem}
\begin{solution}
%%%%%%%%%%%%%%%%%%%%%%%%%%%%%%%%%%%%%%%%%%%%%%%%%%%%
%% Ваше решение задачи здесь
\ensuremath{
\begin{pmatrix}
	1 & 2 & 3\\ 2 & 4 & 6 \\ 3 & 6 & 9\\
\end{pmatrix} \cdot
\begin{pmatrix}
	-1 & -2 & -4 \\ -1 & -2 & -4 \\ 1 & 2 & 4 \\
\end{pmatrix} = 
\begin{pmatrix}
	-1-2+3 \enspace & -2 -4 + 6\enspace & -4 -8 + 12\\
	-2-4+6 \enspace & -4-8+12 \enspace  & -8-16+24\\
	-3-6+9 \enspace & -6-12+18 \enspace & -12 -24 + 36\\
\end{pmatrix} =\\=
\begin{pmatrix}
	0 & 0 & 0\\
	0 & 0 & 0\\
	0 & 0 & 0\\
\end{pmatrix}
}

\end{solution} 

%%%%%%%%%%%%%%%%%%%%%%%%%%%%%%%%%%%%%%%%%%%%%%%%%%%%
% Задача 2
\begin{problem}{220(f)}
$\left(
\begin{array}{rrr}
a & b & c \\
c & b & a \\
1 & 1 & 1 \\
\end{array}
\right)
\cdot
\left(
\begin{array}{rrr}
1 & a & c \\
1 & b & b \\
1 & c & a \\
\end{array}
\right)
$

\end{problem}
\begin{solution}
%%%%%%%%%%%%%%%%%%%%%%%%%%%%%%%%%%%%%%%%%%%%%%%%%%%%
%% Ваше решение задачи здесь
\ensuremath{
\begin{pmatrix}
	a & b & c \\ c & b & a \\ 1 & 1 & 1 \\
\end{pmatrix} \cdot
\begin{pmatrix}
	1 & a & c \\ 1 & b & b \\ 1 & c & a \\
\end{pmatrix} = 
\begin{pmatrix}
	a+b+c & a^2+b^2+c^2 & 2ac+b^2+\\
	a+b+c & 2ac+b^2     & a^2+b^2+c^2\\
	3     & a+b+c       & a+b+c\\
\end{pmatrix}
}\\
\end{solution} 

%%%%%%%%%%%%%%%%%%%%%%%%%%%%%%%%%%%%%%%%%%%%%%%%%%%%
% Задача 3
\begin{problem}{274}
Вычислить определитель $\left| \begin{array}{rrr}246 & 427 & 327 \\ 1014 & 543 & 443 \\ -342 & 721 & 621 \end{array} \right|$
\end{problem}
\begin{solution}
%%%%%%%%%%%%%%%%%%%%%%%%%%%%%%%%%%%%%%%%%%%%%%%%%%%%
%% Ваше решение задачи здесь
\ensuremath{
	\begin{vmatrix}
		246 & 427 & 327 \\ 1014 & 543 & 443 \\ -342 & 721 & 621
	\end{vmatrix} = 6 \cdot
	\begin{vmatrix}
		41  & 427 & 327\\
		169 & 543 & 443\\
		-57 & 721 & 621\\
	\end{vmatrix} = 6 \cdot (41 \cdot 543 \cdot 621 + 427 \cdot 443 \cdot (-57) + 327 \cdot 169 \cdot 721 - (-57) \cdot 543 \cdot 327 - 169 \cdot 427 \cdot 621 - 41 \cdot 721 \cdot 443) = -29 \: 400 \: 000
}\\
\end{solution} 

%%%%%%%%%%%%%%%%%%%%%%%%%%%%%%%%%%%%%%%%%%%%%%%%%%%%
% Задача 4
\newpage
\begin{problem}{221}
Выполнить действия: \\
\noindent
\begin{tabular}{ll}
a) 
$\left(
\begin{array}{rrr}
2 & 1 & 1\\
3 & 1 & 0\\
0 & 1 & 2
\end{array}
\right)^2
$ 
& b)
$\left(
\begin{array}{rr}
2 & 1 \\
1 & 3
\end{array}
\right)^3
$ \\
 & \\
c)
$\left(
\begin{array}{rr}
3 & 2 \\
-4 & -2 \\
\end{array}
\right)^5
$
& d)
$\left(
\begin{array}{rr}
1 & 1 \\
0 & 1 \\
\end{array}
\right)^n
$
\\
\end{tabular}

\end{problem}
\begin{solution}
%%%%%%%%%%%%%%%%%%%%%%%%%%%%%%%%%%%%%%%%%%%%%%%%%%%%
%% Ваше решение задачи здесь

\begin{enumerate}[label=\alph*)]
	\item {
\ensuremath{
	\begin{pmatrix}
		2 & 1 & 1\\
		3 & 1 & 0\\
		0 & 1 & 2
	\end{pmatrix}^2 = 
	\begin{pmatrix}
		4+3   \enspace & 6+1   \enspace & 2+2\\
		6+3   \enspace & 3+1   \enspace & 3\\
		3     \enspace & 1+2   \enspace & 4\\
	\end{pmatrix} = 
	\begin{pmatrix}
		7 & 7 & 4\\
		9 & 4 & 3\\
		3 & 3 & 4\\
	\end{pmatrix}
}
}
	\item{
\ensuremath{
	\begin{pmatrix}
		2 & 1\\
		1 & 3\\
	\end{pmatrix}^3 =
	\begin{pmatrix}
		5 & 5\\
		5 & 10\\
	\end{pmatrix} \cdot
	\begin{pmatrix}
		2 & 1\\
		1 & 3\\
	\end{pmatrix} = 
	\begin{pmatrix}
		15 & 20\\
		20 & 35\\
	\end{pmatrix}
}
}
\item{
\ensuremath{
	\begin{pmatrix}
		3 & 2 \\
		-4 & -2 \\
	\end{pmatrix}^5 = 
	\begin{pmatrix}
		3 & 2 \\
		-4 & -2 \\
	\end{pmatrix} \cdot
	\left(	\begin{pmatrix}
		3 & 2 \\
		-4 & -2 \\
	\end{pmatrix}^2\right)^2=
	\begin{pmatrix}
		3 & 2 \\
		-4 & -2 \\
	\end{pmatrix} \cdot
	\begin{pmatrix}
		1  & 2\\
		-4 & -4\\
	\end{pmatrix}^2=\\=
	\begin{pmatrix}
		3 & 2 \\
		-4 & -2 \\
	\end{pmatrix} \cdot
	\begin{pmatrix}
		-7 & -6\\
		12 & 8\\
	\end{pmatrix} =
	\begin{pmatrix}
		3 & -2\\
		4 & 8\\
	\end{pmatrix}
}
}
\item{
Покажем с помощью метода матиматической индуции, что $
\begin{pmatrix}
	1 & 1\\
	0 & 1\\
\end{pmatrix}^n =
\begin{pmatrix}
	1 & n\\
	0 & 1\\
\end{pmatrix}\\ \forall n \in \mathbb{N}
$\\
\textit{База индукции}(n=1):
$
\begin{pmatrix}
	1 & 1\\
	0 & 1\\
\end{pmatrix}^1=
\begin{pmatrix}
	1 & n\\
	0 & 1\\
\end{pmatrix}$ - выполнено\\
\textit{Индукционный переход: } Пусть для некоторого $k=n \in \mathbb{N}$ выполнено утверждение $a_n=
\begin{pmatrix}
	1 & 1\\
	0 & 1\\
\end{pmatrix}^n =
\begin{pmatrix}
	1 & n\\
	0 & 1\\
\end{pmatrix}$\\
Докажем его для k=n+1: $a_n+1=a_n \cdot 
\begin{pmatrix}
	1 & 1\\
	0 & 1\\
\end{pmatrix} = 
\begin{pmatrix}
	1 & 1\\
	0 & 1\\
\end{pmatrix} \cdot
\begin{pmatrix}
	1 & n\\
	0 & 1\\
\end{pmatrix}=\\=
\begin{pmatrix}
	1 & n+1\\
	0 & 1\\
\end{pmatrix}$ - Выполнено\\
Таким образом: $\begin{pmatrix}
	1 & 1\\
	0 & 1\\
\end{pmatrix}^n =
\begin{pmatrix}
	1 & n\\
	0 & 1\\
\end{pmatrix}$
}
\end{enumerate}
\end{solution}
\newline

%%%%%%%%%%%%%%%%%%%%%%%%%%%%%%%%%%%%%%%%%%%%%%%%%%%%
% Задача 5
\begin{problem}{224}
Вычислить $A\cdot A'$, где $A=\left( \begin{array}{cccc}3 & 2 & 1& 2\\4 & 1 & 1 & 3 \end{array} \right)$
\end{problem}
\begin{solution}
%%%%%%%%%%%%%%%%%%%%%%%%%%%%%%%%%%%%%%%%%%%%%%%%%%%%
%% Ваше решение задачи здесь
\ensuremath{
A\cdot A' = 
\begin{pmatrix}
	3 & 2 & 1 & 2\\
	4 & 1 & 1 & 3\\
\end{pmatrix} \cdot
\begin{pmatrix}
	3 & 4\\
	2 & 1\\
	1 & 1\\
	2 & 3\\
\end{pmatrix}=
\begin{pmatrix}
	18 & 21\\
	21 & 27\\
\end{pmatrix}
}

\end{solution} 

%%%%%%%%%%%%%%%%%%%%%%%%%%%%%%%%%%%%%%%%%%%%%%%%%%%%
% Задача 6
\begin{problem}{232(c)}
Вычислить определитель:
c) $\left| \begin{array}{rrr}a & a & a \\ -a & a & x\\-a & -a & x \end{array} \right|$

\end{problem}
\begin{solution}
%%%%%%%%%%%%%%%%%%%%%%%%%%%%%%%%%%%%%%%%%%%%%%%%%%%%
%% Ваше решение задачи здесь
$
\begin{vmatrix}
	a & a & a\\
	-a & a & x\\
	-a & -a & x\\
\end{vmatrix} \:
\underset{\begin{subarray}{l}\text{$S_2+S_1$}\\
		\text{$S_3+S_1$}\end{subarray}}{=} \:
\begin{vmatrix}
	a & a  & a\\
	0 & 2a & a + x\\
	0 & 0  & a + x\\
\end{vmatrix} = a \cdot 2a \cdot (a+x) = 2a^2(a+x)
$\vspace{.1cm}\\
\end{solution} 

%%%%%%%%%%%%%%%%%%%%%%%%%%%%%%%%%%%%%%%%%%%%%%%%%%%%
% Задача 7
\begin{problem}{232(d)}
Вычислить определитель:
d) $\left| \begin{array}{rrr}1 & 1 & 1 \\ 1 & 2 & 3 \\ 1 & 3 & 6 \end{array} \right|$ 

\end{problem}
\begin{solution}
%%%%%%%%%%%%%%%%%%%%%%%%%%%%%%%%%%%%%%%%%%%%%%%%%%%%
%% Ваше решение задачи здесь
$
\begin{vmatrix}
	1 & 1 & 1\\
	1 & 2 & 3\\
	1 & 3 & 6\\
\end{vmatrix} \:
\underset{\begin{subarray}{l}{\text{$S_2-S_1$}}\\
	\text{$S_3-S_1$}\end{subarray}}{=} \:
\begin{vmatrix}
	1 & 1 & 1\\
	0 & 1 & 2\\
	0 & 2 & 5
\end{vmatrix} =
\begin{vmatrix}
	1 & 2\\
	2 & 5\\
\end{vmatrix}=5-4=1
$

\end{solution} 

%%%%%%%%%%%%%%%%%%%%%%%%%%%%%%%%%%%%%%%%%%%%%%%%%%%%
% Задача 8
\begin{problem}{232(f)}
Вычислить определитель:
f) $\left| \begin{array}{rrr}1 & 1 & 1 \\ 1 & \omega & \omega\\1 & \omega^2 & \omega \end{array} \right|$
\end{problem}
\begin{solution}
%%%%%%%%%%%%%%%%%%%%%%%%%%%%%%%%%%%%%%%%%%%%%%%%%%%%
%% Ваше решение задачи здесь
$
\begin{vmatrix}
	1 & 1 & 1 \\
	 1 & \omega & \omega\\
	 1 & \omega^2 & \omega\\
\end{vmatrix} \:
\underset{\begin{subarray}{l}{\text{$S_2-S_1$}}\\
	\text{$S_3-S_1$}\end{subarray}}{=} \:
\begin{vmatrix}
	1 & 1 & 1\\
	0 & \omega-1 & \omega-1\\
	0 & \omega^2-1 & \omega-1
\end{vmatrix}=
\begin{vmatrix}
	\omega-1 & \omega-1\\
	\omega^2-1 & \omega-1
\end{vmatrix}=\vspace{.1cm}\\=(\omega-1)^2-(\omega-1) \cdot (\omega^2-1) = -\omega + 2 \omega^2 - \omega^3
$\\
\end{solution} 

%%%%%%%%%%%%%%%%%%%%%%%%%%%%%%%%%%%%%%%%%%%%%%%%%%%%
% Задача 9
\newpage
\begin{problem}{284}
$\left| \begin{array}{rrr}x & y & x+y \\ y & x+y & x \\ x+y & x & y \end{array} \right|$
\end{problem}
\begin{solution}
%%%%%%%%%%%%%%%%%%%%%%%%%%%%%%%%%%%%%%%%%%%%%%%%%%%%
%% Ваше решение задачи здесь
$
\begin{vmatrix}
	x   & y   & x+y\\
	y   & x+y & x\\
	x+y & x   & y\\
\end{vmatrix}=3xy(x+y)-(x+y)^3-x^3-y^3=-2(x^3+y^3)
$\\

\end{solution} 
