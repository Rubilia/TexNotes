\documentclass[a4paper, 12pt]{article}
\usepackage[T2A]{fontenc}% Внутренняя T2A кодировка TeX
\usepackage[utf8]{inputenc}% кодировка файла
\usepackage[russian]{babel}% поддержка переносов в русском языке
\usepackage{comment}% enables the use of multi-line comments (\ifx \fi) 
\usepackage{fullpage}% changes the margin
\usepackage[a4paper, total={7in, 10in}]{geometry}
\usepackage[fleqn]{amsmath}
\usepackage{amssymb,amsthm}  
\usepackage{graphicx}
\usepackage{verbatim}
\usepackage{float}
\usepackage{xcolor}
\usepackage{mdframed}
\usepackage[shortlabels]{enumitem}
\usepackage{indentfirst}
\usepackage{hyperref}
\usepackage{cancel}


\renewcommand{\thesubsection}{\thesection.\alph{subsection}}

\newenvironment{problem}[2][Задача]
{ \begin{mdframed}[backgroundcolor=gray!10] \textbf{#1 #2.} \\}
	{  \end{mdframed}}

\newenvironment{solution}
{\textit{Решение:}\vspace{.1cm}\\}
{\vspace{.1cm}\noindent\rule{7in}{1.5pt}}

\renewcommand{\qed}{\quad\qedsymbol}

\begin{document}
\noindent
\large\textbf{Домашняя работа №9} \hfill  Дата: 28.10.2021  \\
\noindent\rule{7in}{2pt}

%%%%%%%%%%%%%%%%%%%%%%%%%%%%%%%%%%%%%%%%%%%%%%%%%%%%
% Задача 1
\begin{problem}{410(d)}
Обратить матрицу:
$\left(\begin{array}{rrrr}1 & 3 & -5 & 7\\0 & 1 & 2 & -3\\ 0 & 0 & 1 & 2 \\ 0 & 0 & 0 & 1\end{array}\right)$

\end{problem}
\begin{solution}
%%%%%%%%%%%%%%%%%%%%%%%%%%%%%%%%%%%%%%%%%%%%%%%%%%%%
%% Ваше решение задачи здесь
$
\left(
\begin{array}{cccc|cccc}
	1 & 3 & -5 & 7 & 1 & 0 & 0 & 0\\
	0 & 1 & 2 & -3 & 0 & 1 & 0 & 0\\ 
	0 & 0 & 1 & 2  & 0 & 0 & 1 & 0\\ 
	0 & 0 & 0 & 1  & 0 & 0 & 0 & 1\\
\end{array}
\right) \: 
\underset{
\begin{subarray}{c}
	\text{$R_1: R_1-3R_2$}\\
	\text{$R_2: R_2-3R_3$}\\
	\text{$R_3: R_3-2R_4$}\\
\end{subarray}
}{\longrightarrow} \:
\left(
\begin{array}{cccc|cccc}
	1 & 0 & -11 & 16 & 1 & -3 & 0 & 0\\
	0 & 1 & 0 & -7   & 0 & 1 & -2 & 0\\ 
	0 & 0 & 1 & 0    & 0 & 0 & 1 & -2\\ 
	0 & 0 & 0 & 1    & 0 & 0 & 0 & 1\\
\end{array}
\right)  \rightarrow \\
\underset{
	\begin{subarray}{c}
		\text{$R_1: R_1+11R_3$}\\
		\text{$R_2: R_2+7R_4$}\\
	\end{subarray}
}{\longrightarrow} \:
\left(
\begin{array}{cccc|cccc}
	1 & 0 & 0 & 16 & 1 & -3 & 11 & -22\\
	0 & 1 & 0 & 0   & 0 & 1 & -2 & 7\\ 
	0 & 0 & 1 & 0    & 0 & 0 & 1 & -2\\ 
	0 & 0 & 0 & 1    & 0 & 0 & 0 & 1\\
\end{array}
\right) \: 
\underset{R_1: R_1-16R_4
}{\longrightarrow} \:
\left(
\begin{array}{cccc|cccc}
	1 & 0 & 0 & 0   & 1 & -3 & 11 & -38\\
	0 & 1 & 0 & 0   & 0 & 1 & -2 & 7\\ 
	0 & 0 & 1 & 0   & 0 & 0 & 1 & -2\\ 
	0 & 0 & 0 & 1   & 0 & 0 & 0 & 1\\
\end{array}
\right) 
$\\
$
\begin{pmatrix}
	1 & 3 & -5 & 7\\
	0 & 1 & 2 & -3\\ 
	0 & 0 & 1 & 2 \\ 
	0 & 0 & 0 & 1
\end{pmatrix}^{-1} =
\begin{pmatrix}
	1 & -3 & 11 & -38\\
	0 & 1 & -2 & 7\\
	0 & 0 & 1 & -2\\
	0 & 0 & 0 & 1\\
\end{pmatrix}
$\vspace{.2cm}\\
\end{solution} 

%%%%%%%%%%%%%%%%%%%%%%%%%%%%%%%%%%%%%%%%%%%%%%%%%%%%
% Задача 2
\begin{problem}{410(f)}
Обратить матрицу:
$\left(\begin{array}{rrrr}1 & 1 & 1 & 1\\1 & 1 & -1 & -1\\ 1 & -1 & 1 & -1 \\ 1 & -1 & -1 & 1\end{array}\right)$

\end{problem}
\begin{solution}
%%%%%%%%%%%%%%%%%%%%%%%%%%%%%%%%%%%%%%%%%%%%%%%%%%%%
%% Ваше решение задачи здесь
\small
$
\left(
\begin{array}{cccc|cccc}
	1 & 1 & 1 & 1 &    1 & 0 & 0 & 0\\
	1 & 1 & -1 & -1&   0 & 1 & 0 & 0\\
	1 & -1 & 1 & -1&   0 & 0 & 1 & 0\\
	1 & -1 & -1 & 1&   0 & 0 & 0 & 1\\
\end{array}
\right)  \: 
\underset{
	\begin{subarray}{c}
		\text{$R_2: R_2-R_3$}\\
		\text{$R_3: R_3-R_4$}\\
		\text{$R_4: R_4-R_1$}\\
	\end{subarray}
}{\longrightarrow} \:
\left(
\begin{array}{cccc|cccc}
	1 & 1 & 1 & 1 &    1 & 0 & 0 & 0\\
	0 & 2 & -2 & 0 &   0 & 1 & -1 & 0\\
	0 & 0 & 2 & -2 &   0 & 0 & 1 & -1\\
	0 & -2 & -2 & 0&   -1 & 0 & 0 & 1\\
\end{array}
\right) \rightarrow\\
\underset{R_4: R_4+R_2+2R_3
}{\longrightarrow} \:
\left(
\begin{array}{cccc|cccc}
	1 & 1 & 1 & 1 &    1 & 0 & 0 & 0\\
	0 & 2 & -2 & 0 &   0 & 1 & -1 & 0\\
	0 & 0 & 2 & -2 &   0 & 0 & 1 & -1\\
	0 & 0 & 0 & -4 &   -1 & 1 & 1 & -1\\
\end{array}
\right)  \: 
\underset{
	\begin{subarray}{c}
		\text{$R_2: \frac{R_2}{2}$}\\
		\text{$R_3: \frac{R_3}{2}$}\\
		\text{$R_4: -\frac{R_4}{4}$}\\
	\end{subarray}
}{\longrightarrow} \:
\left(
\begin{array}{cccc|cccc}
	1 & 1 & 1 & 1 &    1 & 0 & 0 & 0\\
	0 & 1 & -1 & 0 &   0 & \frac{1}{2} & -\frac{1}{2} & 0\\
	0 & 0 & 1 & -1 &   0 & 0 & \frac{1}{2} & -\frac{1}{2}\\
	0 & 0 & 0 & 1 &   \frac{1}{4} & -\frac{1}{4} & -\frac{1}{4} & \frac{1}{4}\\
\end{array}
\right) \rightarrow\\
\underset{
	\begin{subarray}{c}
		\text{$R_1: R_1-R_2-2R_3-R_4$}\\
		\text{$R_2: R_2+R_3+R_4$}\\
		\text{$R_3: R_3+R_4$}\\
	\end{subarray}
}{\longrightarrow} \:
\left(
\begin{array}{cccc|cccc}
	1 & 0 & 0 & 0 &   \frac{1}{4} & \frac{1}{4} & \frac{1}{4} & \frac{1}{4}\\
	0 & 1 & 0 & 0 &   \frac{1}{4} & \frac{1}{4} & -\frac{1}{4} & -\frac{1}{4}\\
	0 & 0 & 1 & 0 &   \frac{1}{4} & -\frac{1}{4} & \frac{1}{4} & -\frac{1}{4}\\
	0 & 0 & 0 & 1 &   \frac{1}{4} & -\frac{1}{4} & -\frac{1}{4} & \frac{1}{4}\\
\end{array}
\right)\\
\left(
\begin{array}{cccc}
	1 & 1 & 1 & 1\\
	1 & 1 & -1 & -1\\
	1 & -1 & 1 & -1\\
	1 & -1 & -1 & 1\\
\end{array}
\right)^{-1} = \frac{1}{4} \cdot
\left(
\begin{array}{cccc}
	1 & 1 & 1 & 1\\
	1 & 1 & -1 & -1\\
	1 & -1 & 1 & -1\\
	1 & -1 & -1 & 1\\
\end{array}
\right)\vspace{.2cm}\\
$


\end{solution} 

%%%%%%%%%%%%%%%%%%%%%%%%%%%%%%%%%%%%%%%%%%%%%%%%%%%%
% Задача 3
\begin{problem}{416}
$\left( \begin{array}{ccccc}2 & -1 & 0 & \ldots & 0 \\ -1 & 2 & -1 & \ldots & 0 \\ 0 & -1 & 2 & \ldots & 0 \\ \vdots & \vdots & \vdots & \ddots & \vdots \\ 0 & 0 & 0 & \ldots & -2 \end{array} \right)^{-1}$

\end{problem}
\begin{solution}
%%%%%%%%%%%%%%%%%%%%%%%%%%%%%%%%%%%%%%%%%%%%%%%%%%%%
%% Ваше решение задачи здесь


\end{solution} 

%%%%%%%%%%%%%%%%%%%%%%%%%%%%%%%%%%%%%%%%%%%%%%%%%%%%
% Задача 4
\begin{problem}{442(b)}
Найти ранг матрицы:
$\left(\begin{array}{rrrr}2 & 1 & 11 & 2\\1 & 0 & 4 & -1\\ 11 & 4 & 56 & 5 \\ 2 & -1 & 5 & -6\end{array}\right)$

\end{problem}
\begin{solution}
%%%%%%%%%%%%%%%%%%%%%%%%%%%%%%%%%%%%%%%%%%%%%%%%%%%%
%% Ваше решение задачи здесь
$
\begin{pmatrix}
	2 & 1 & 11 & 2\\
	1 & 0 & 4 & -1\\ 
	11 & 4 & 56 & 5 \\ 
	2 & -1 & 5 & -6
\end{pmatrix} \:
\underset{
\begin{subarray}{c}
	\text{$R_1: R_1-2R_2$}\\
	\text{$R_3: R_3-11R_2$}\\
	\text{$R_4: R_3-2R_2$}\\
\end{subarray}
}{\longrightarrow} \:
\begin{pmatrix}
	0 & 1 & 3 & 4\\
	1 & 0 & 4 & -1\\
	0 & 4 & 12 & 16\\
	0 & -1 & -3 & -4\\
\end{pmatrix} \:
\underset{
\begin{subarray}{c}
	\text{$R_3: R_3-4R_1$}\\
	\text{$R_4: R_4+R_1$}\\
	\text{$R_1 \leftrightarrow R_2$}\\
\end{subarray}
}{\longrightarrow} \:
\begin{pmatrix}
	1 & 0 & 4 & -1\\
	0 & 1 & 3 & 4\\
	0 & 0 & 0 & 0\\
	0 & 0 & 0 & 0\\
\end{pmatrix}
$\\

Ранг данной маттрицы: 2\\
\end{solution} 

%%%%%%%%%%%%%%%%%%%%%%%%%%%%%%%%%%%%%%%%%%%%%%%%%%%%
% Задача 5
\begin{problem}{442(с)}
Найти ранг матрицы:
$\left(\begin{array}{rrrrr}1 & 0 & 0 & 1 & 4\\0 & 1 & 0 & 2 & 5\\ 0 & 0 & 1 & 3 & 6 \\ 1 & 2 & 3 & 14 & 32\\ 4 & 5 & 6 & 32 & 77\end{array}\right)$

\end{problem}
\begin{solution}
%%%%%%%%%%%%%%%%%%%%%%%%%%%%%%%%%%%%%%%%%%%%%%%%%%%%
%% Ваше решение задачи здесь

$
\begin{pmatrix}
	1 & 0 & 0 & 1 & 4\\
	0 & 1 & 0 & 2 & 5\\ 
	0 & 0 & 1 & 3 & 6 \\ 
	1 & 2 & 3 & 14 & 32\\ 
	4 & 5 & 6 & 32 & 77\\
\end{pmatrix} \:
\underset{
	\begin{subarray}{c}
		\text{$R_4: R_4-R_1-2R_2-3R_3$}\\
		\text{$R_5: R_5-4R_1-5R_2-6R_3$}\\
	\end{subarray}
}{\longrightarrow} \:
\begin{pmatrix}
	1 & 0 & 0 & 1 & 4\\
	0 & 1 & 0 & 2 & 5\\
	0 & 0 & 1 & 3 & 6\\
	0 & 0 & 0 & 0 & 0\\
	0 & 0 & 0 & 0 & 0\\
\end{pmatrix}\\
$
Ранг данной маттрицы: 3\\

\end{solution} 

%%%%%%%%%%%%%%%%%%%%%%%%%%%%%%%%%%%%%%%%%%%%%%%%%%%%
% Задача 6
\begin{problem}{1033(b)}
Найти собственные значения матрицы\\
$\left( \begin{array}{rrrrrrr}0 & 1 & 0 & \ldots & 0 & 0 & 0\\ -1 & 0 & 1 & \ldots & 0 & 0 & 0 \\ 0 & -1 & 0 & \ldots & 0 & 0 & 0 \\ \vdots & \vdots & \vdots & \ddots & \vdots & \vdots & \vdots \\ 0 & 0 & 0 & \ldots & 0 & 1 & 0 \\ 0 & 0 & 0 & \ldots & -1 & 0 & 1 \\ 0 & 0 & 0 & \ldots & 0 & -1 & 0 \end{array} \right)$

\end{problem}
\begin{solution}
%%%%%%%%%%%%%%%%%%%%%%%%%%%%%%%%%%%%%%%%%%%%%%%%%%%%
%% Ваше решение задачи здесь
$det(A-\lambda E)=0 \leftrightarrow \: \Delta_n =
\begin{vmatrix}
	-\lambda & 1 & 0 & \ldots & 0 & 0 & 0\\ 
	-1 & -\lambda & 1 & \ldots & 0 & 0 & 0 \\ 
	0 & -1 & -\lambda& \ldots & 0 & 0 & 0 \\ 
	\vdots & \vdots & \vdots & \ddots & \vdots & \vdots & \vdots \\ 
	0 & 0 & 0 & \ldots & -\lambda & 1 & 0 \\ 
	0 & 0 & 0 & \ldots & -1 & -\lambda & 1 \\ 
	0 & 0 & 0 & \ldots & 0 & -1 & -\lambda \\
\end{vmatrix} =
-\lambda \cdot
\underbrace{
\begin{vmatrix}
	\lambda & 1 & \ldots & 0\\ 
	-1 & -\lambda & \ldots & 0 \\ 
	0 & -1 & \ldots & 0\\ 
	\vdots & \vdots &\ddots & \vdots\\ 
	0 & 0 & \ldots & -\lambda \\ 
\end{vmatrix}
}_{\Delta_{n-1}} -\\-
\begin{vmatrix}
		-\lambda & 1 & 0 & \ldots & 0 & 0\\ 
	-1 & -\lambda & 1 & \ldots & 0 & 0\\ 
	0 & -1 & -\lambda& \ldots & 0 & 0\\ 
	\vdots & \vdots & \vdots & \ddots & \vdots & \vdots\\ 
	0 & 0 & 0 & \ldots & -\lambda & 1\\ 
	0 & 0 & 0 & \ldots & -1 & -\lambda\\ 
	0 & 0 & 0 & \ldots & 0 & -1\\
\end{vmatrix} = -\lambda \cdot \Delta_{n-1} + 
\underbrace{
\begin{vmatrix}
	\lambda & 1 & \ldots & 0\\ 
	-1 & -\lambda & \ldots & 0 \\ 
	0 & -1 & \ldots & 0\\ 
	\vdots & \vdots &\ddots & \vdots\\ 
	0 & 0 & \ldots & -\lambda \\ 
\end{vmatrix}
}_{\Delta_{n-2}} = -\lambda \cdot \Delta_{n-1} + \Delta_{n-2}\vspace{.2cm}\\
G(z) = \Delta_0 + z \cdot \Delta_1 + z^2 \cdot \Delta_2 + \ldots \text{- производящая функция последовательности $\{\Delta_i\}$}\\
G(z) = \frac{1+\lambda z - \lambda z}{1+\lambda z -z^2} = \frac{1}{1+\lambda z -z^2}\\
z = cos(\phi) +isin(\phi)\\
1 + \lambda z -z^2 = 0 \implies \lambda = z - \frac{1}{z} = cos(\phi) +isin(\phi) - (cos(\phi) -isin(\phi)) =2i sin(\phi)\\
z_{1,2}=\frac{-\lambda \pm \sqrt{\lambda^2+4}}{-2}=cos(\phi) \pm isin(\phi)\\
\frac{A}{z_1-z} + \frac{B}{z_2-z} = \frac{1}{1+\lambda z -z^2} \leftrightarrow
A(z_2-z) + B(z_1-z)=1 \implies
\begin{cases}
	z=z_1: \: A(z_2-z_1)=1\\
	z=z_2: \: B(z_1-z_2)=1\\
\end{cases} \leftrightarrow\\\leftrightarrow
\begin{cases}
	A=\frac{1}{-2isin(\phi)}=\frac{i}{2sin(\phi)}\\
	B = \frac{1}{2isin(\phi)}=-\frac{i}{2sin(\phi)}
\end{cases}\\$

$
G(z) = \frac{\frac{A}{z_1}}{1-\frac{z}{z_1}} + \frac{\frac{B}{z_2}}{1-\frac{z}{z_2}}=
\frac{i}{2sin(\phi) \cdot z_1}\cdot \sum\limits_{n=0}^{+\infty} \left[z^n \cdot (\frac{1}{z_1})^n\right] + \frac{-i}{2sin(\phi) \cdot z_2} \cdot
\sum\limits_{n=0}^{+\infty} \left[z^n \cdot (\frac{1}{z_2})^n\right]=\\=
\sum\limits_{n=0}^{+\infty} \left[\frac{i}{2sin(\phi)} z^n \cdot((cos(\phi) + isin(\phi))^{-(n+1)} - (cos(\phi) - isin(\phi))^{-(n+1)})\right] =\\ =
\sum\limits_{n=0}^{+\infty} \left[z^n \cdot \frac{i}{2sin(\phi)} \cdot (\cancel{cos((n+1)\phi)} -i sin((n+1)\phi) - \cancel{cos((n+1)\phi)} -isin((n+1)\phi)) \right] =\\=
\sum\limits_{n=0}^{+\infty} \left[z^n \cdot \frac{sin((n+1)\phi)}{sin(\phi)}\right] \implies \Delta_n = \frac{sin((n+1)\phi)}{sin(\phi)}\\
\Delta_n = 0 \leftrightarrow sin((n+1) \phi) = 0 \leftrightarrow \phi = \frac{\pi k}{n+1}, \: k \in \mathbb{Z} \implies
\lambda = 2 i sin(\phi) = 2i sin(\frac{\pi k}{n+1})
$


\end{solution} 

%%%%%%%%%%%%%%%%%%%%%%%%%%%%%%%%%%%%%%%%%%%%%%%%%%%%
% Задача 7
\begin{problem}{1034}
Найти собственные значения матрицы\\
$\left( \begin{array}{rrrrrr}-1 & 1 & 0 & \ldots & 0 & 0 \\ 1 & 0 & 1 & \ldots & 0 & 0 \\ 0 & 1 & 0 & \ldots & 0 & 0 \\ \vdots & \vdots & \vdots & \ddots & \vdots & \vdots \\ 0 & 0 & 0 & \ldots & 0 & 1 \\ 0 & 0 & 0 & \ldots & 1 & 0 \end{array} \right)$

\end{problem}
\begin{solution}
%%%%%%%%%%%%%%%%%%%%%%%%%%%%%%%%%%%%%%%%%%%%%%%%%%%%
%% Ваше решение задачи здесь
$
det(A-\lambda E) = 0 \leftrightarrow \Delta_n =
\begin{vmatrix}
	-1-\lambda & 1 & 0 & \ldots & 0 & 0 \\ 
	1 & -\lambda & 1 & \ldots & 0 & 0 \\ 
	0 & 1 & -\lambda & \ldots & 0 & 0 \\ 
	\vdots & \vdots & \vdots & \ddots & \vdots & \vdots 
	\\ 0 & 0 & 0 & \ldots & -\lambda & 1 \\ 
	0 & 0 & 0 & \ldots & 1 & -\lambda\\
\end{vmatrix}= 0\\
$

$
\Delta_n = -\lambda \cdot
\underbrace{
\begin{vmatrix}
	-1-\lambda & 1 & 0 & \ldots & 0 & 0 \\ 
	1 & -\lambda & 1 & \ldots & 0 & 0 \\ 
	0 & 1 & -\lambda & \ldots & 0 & 0 \\ 
	\vdots & \vdots & \vdots & \ddots & \vdots & \vdots 
	\\ 0 & 0 & 0 & \ldots & -\lambda & 1 \\ 
	0 & 0 & 0 & \ldots & 1 & -\lambda\\
\end{vmatrix}
}_{\Delta_{n-1}} -
\begin{vmatrix}
	-1-\lambda & 1 & 0 & \ldots & 0 & 0 & 0 \\ 
	1 & -\lambda & 1 & \ldots & 0 & 0 & 0 \\ 
	0 & 1 & -\lambda & \ldots & 0 & 0 & 0 \\ 
	\vdots & \vdots & \vdots & \ddots & \vdots & \vdots & \vdots\\
	0 & 0 & 0 & \ldots & 1 & -\lambda & 1 \\ 
	0 & 0 & 0 & \ldots & 0 & 0 & 1\\
\end{vmatrix} =\\= -\lambda \Delta_{n-1} -
\underbrace{
\begin{vmatrix}
	-1-\lambda & 1 & 0 & \ldots & 0 & 0 \\ 
	1 & -\lambda & 1 & \ldots & 0 & 0 \\ 
	0 & 1 & -\lambda & \ldots & 0 & 0 \\ 
	\vdots & \vdots & \vdots & \ddots & \vdots & \vdots 
	\\ 0 & 0 & 0 & \ldots & -\lambda & 1 \\ 
	0 & 0 & 0 & \ldots & 1 & -\lambda\\
\end{vmatrix}
}_{\Delta_{n-2}} = -\lambda \Delta_{n-1} - \Delta_{n-2}\\
G(z) = \Delta_0 + \Delta_1 z + \Delta_2 z^2 + \ldots$ - производящая функция последовательности $\{\Delta_i\}$\\
$
G(z) = \frac{1+z(1-\lambda+\lambda)}{1+\lambda z + z^2}=\frac{1-z}{1+\lambda z +z^2}\\
z = cos(\phi) + isin(\phi)\\
1+\lambda z +z^2=0 \implies \lambda = -(z+\frac{1}{z})=-(cos(\phi) + isin(\phi) + cos(\phi) -isin(\phi))=\\=-2cos(\phi)\\
z_{1,2}=\frac{-\lambda \pm \sqrt{\lambda^2-4}}{2}=\frac{2cos(\phi) \pm 2isin(\phi)}{2}=cos(\phi) \pm sin(\phi)\\
G(z) = \frac{A}{z_1-z} + \frac{B}{z_2-z} = \frac{1-z}{1+\lambda z + z^2} \leftrightarrow A(z_2-z)+B(z_1-z)=1-z \leftrightarrow\\ \leftrightarrow
\begin{cases}
	z=z_1: \: A(z_2-z_1)=1-z_1\\
	z=z_2: \: B(z_1-z_2)=1-z_2\\
\end{cases} \leftrightarrow
\begin{cases}
	A = \frac{cos(\phi)-1 + isin(\phi)}{2isin(\phi)}\\
	B = \frac{1-cos(\phi) + isin(\phi)}{2isin(\phi)}\\
\end{cases}
\\$
\end{solution} 

%%%%%%%%%%%%%%%%%%%%%%%%%%%%%%%%%%%%%%%%%%%%%%%%%%%%
% Задача 8
\begin{problem}{1035}
Найти собственные значения матрицы\\
$\left( \begin{array}{rrrrr}0 & x & x & \ldots & x \\ y & 0 & x & \ldots & x \\ y & y & 0 & \ldots & x \\ \vdots & \vdots & \vdots & \ddots & \vdots \\ y & y & y & \ldots & 0 \end{array} \right)$

\end{problem}
\begin{solution}
%%%%%%%%%%%%%%%%%%%%%%%%%%%%%%%%%%%%%%%%%%%%%%%%%%%%
%% Ваше решение задачи здесь


\end{solution} 

\end{document}