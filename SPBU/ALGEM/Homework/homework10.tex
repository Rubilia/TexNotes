\documentclass[a4paper, 12pt]{article}
\usepackage[T2A]{fontenc}% Внутренняя T2A кодировка TeX
\usepackage[utf8]{inputenc}% кодировка файла
\usepackage[russian]{babel}% поддержка переносов в русском языке
\usepackage{comment}% enables the use of multi-line comments (\ifx \fi) 
\usepackage{fullpage}% changes the margin
\usepackage[a4paper, total={7in, 10in}]{geometry}
\usepackage[fleqn]{amsmath}
\usepackage{amssymb,amsthm}  
\usepackage{graphicx}
\usepackage{verbatim}
\usepackage{float}
\usepackage{xcolor}
\usepackage{mdframed}
\usepackage[shortlabels]{enumitem}
\usepackage{indentfirst}
\usepackage{hyperref}
\usepackage{cancel}

\renewcommand{\thesubsection}{\thesection.\alph{subsection}}

\newenvironment{problem}[2][Задача]
{ \begin{mdframed}[backgroundcolor=gray!10] \textbf{#1 #2.} \\}
	{  \end{mdframed}}

\newenvironment{solution}
{\textit{Решение:}\vspace{.1cm}\\}
{\vspace{.1cm}\noindent\rule{7in}{1.5pt}}

\renewcommand{\qed}{\quad\qedsymbol}

\begin{document}

\noindent
\large\textbf{Домашняя работа №10} \hfill  Дата: XX.YY.2021  \\
\noindent\rule{7in}{2pt}

%%%%%%%%%%%%%%%%%%%%%%%%%%%%%%%%%%%%%%%%%%%%%%%%%%%%
% Задача 1
\begin{problem}{400(e)}
Решить систему уравнений:\\
$\left\{\begin{array}{l}
x_1+x_2+x_3+3x_4=1\\
3x_1-x_2-x_3-2x_4=-4\\
2x_1+3x_2-x_3-x_4=-6\\
x_1+2x_2+3x_3-x_4=-4
\end{array}\right.$

\end{problem}
\begin{solution}
%%%%%%%%%%%%%%%%%%%%%%%%%%%%%%%%%%%%%%%%%%%%%%%%%%%%
%% Ваше решение задачи здесь


\end{solution} 

%%%%%%%%%%%%%%%%%%%%%%%%%%%%%%%%%%%%%%%%%%%%%%%%%%%%
% Задача 2
\begin{problem}{400(f)}
Решить систему уравнений:\\
$\left\{\begin{array}{l}
x_1+2x_2+3x_3-2x_4=6\\
2x_1-x_2-2x_3-3x_4=8\\ 
3x_1+2x_2-x_3+2x_4=4\\
2x_1-3x_2+2x_3+x_4=-8
\end{array}\right.$

\end{problem}
\begin{solution}
%%%%%%%%%%%%%%%%%%%%%%%%%%%%%%%%%%%%%%%%%%%%%%%%%%%%
%% Ваше решение задачи здесь


\end{solution} 

%%%%%%%%%%%%%%%%%%%%%%%%%%%%%%%%%%%%%%%%%%%%%%%%%%%%
% Задача 3
\begin{problem}{443(a)}
Решить систему уравнений:\\
$\left\{\begin{array}{l}
x_1+x_2+x_3+x_4+x_5=0\\
x_1-x_2+2x_3-2x_4+3x_5=0\\
x_1+x_2+4x_3+4x_4+9x_5=0\\
x_1-x_2+8x_3-8x_4+27x_5=0\\
x_1+x_2+16x_3+16x_4+81x_5=0
\end{array}\right.$

\end{problem}
\begin{solution}
%%%%%%%%%%%%%%%%%%%%%%%%%%%%%%%%%%%%%%%%%%%%%%%%%%%%
%% Ваше решение задачи здесь


\end{solution} 

%%%%%%%%%%%%%%%%%%%%%%%%%%%%%%%%%%%%%%%%%%%%%%%%%%%%
% Задача 4
\begin{problem}{443(b)}
Решить систему уравнений:\\
$\left\{\begin{array}{l}
x_1+3x_2+4x_3\\
2x_1-x_2+3x_3=0\\
3x_1-5x_2+4x_3=0\\
x_1+17x_2+4x_3
\end{array}\right.$

\end{problem}
\begin{solution}
%%%%%%%%%%%%%%%%%%%%%%%%%%%%%%%%%%%%%%%%%%%%%%%%%%%%
%% Ваше решение задачи здесь


\end{solution} 

%%%%%%%%%%%%%%%%%%%%%%%%%%%%%%%%%%%%%%%%%%%%%%%%%%%%
% Задача 5
\begin{problem}{449(c}
Выписать фундаментальную систему решений:\\
$\left\{\begin{array}{l}
3x_1+2x_2+x_3=0\\
3x_2+2x_3+x_4=0\\
3x_1-4x_2-3x_3-2x_4=0
\end{array}\right.$

\end{problem}
\begin{solution}
%%%%%%%%%%%%%%%%%%%%%%%%%%%%%%%%%%%%%%%%%%%%%%%%%%%%
%% Ваше решение задачи здесь


\end{solution} 

%%%%%%%%%%%%%%%%%%%%%%%%%%%%%%%%%%%%%%%%%%%%%%%%%%%%
% Задача 6
\begin{problem}{449(d)}
Выписать фундаментальную систему решений:\\
$\left\{\begin{array}{l}
x_1+x_2+x_3+2x_4=0
\end{array}\right.$

\end{problem}
\begin{solution}
%%%%%%%%%%%%%%%%%%%%%%%%%%%%%%%%%%%%%%%%%%%%%%%%%%%%
%% Ваше решение задачи здесь


\end{solution} 
\end{document}