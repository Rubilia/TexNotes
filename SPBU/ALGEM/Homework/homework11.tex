\documentclass[a4paper, 12pt]{article}
\usepackage[T2A]{fontenc}% Внутренняя T2A кодировка TeX
\usepackage[utf8]{inputenc}% кодировка файла
\usepackage[russian]{babel}% поддержка переносов в русском языке
\usepackage{comment}% enables the use of multi-line comments (\ifx \fi) 
\usepackage{fullpage}% changes the margin
\usepackage[a4paper, total={7in, 10in}]{geometry}
\usepackage[fleqn]{amsmath}
\usepackage{amssymb,amsthm}  
\usepackage{graphicx}
\usepackage{verbatim}
\usepackage{float}
\usepackage{xcolor}
\usepackage{mdframed}
\usepackage[shortlabels]{enumitem}
\usepackage{indentfirst}
\usepackage{hyperref}
\usepackage{cancel}

\renewcommand{\thesubsection}{\thesection.\alph{subsection}}

\newenvironment{problem}[2][Задача]
{ \begin{mdframed}[backgroundcolor=gray!10] \textbf{#1 #2.} \\}
	{  \end{mdframed}}

\newenvironment{solution}
{\textit{Решение:}\vspace{.1cm}\\}
{\vspace{.1cm}\noindent\rule{7in}{1.5pt}}

\renewcommand{\qed}{\quad\qedsymbol}

\begin{document}
\noindent
\large\textbf{Домашняя работа №11} \hfill  Дата: XX.YY.2021  \\
\noindent\rule{7in}{2pt}

%%%%%%%%%%%%%%%%%%%%%%%%%%%%%%%%%%%%%%%%%%%%%%%%%%%%
% Задача 1
\begin{problem}{10}
Дан четырёхугольник $ABCD$. Найти такую точку $M$, чтобы 
$\vec{MA}+\vec{MB}+\vec{MC}+\vec{MD}=\vec{0}$.

\end{problem}
\begin{solution}
%%%%%%%%%%%%%%%%%%%%%%%%%%%%%%%%%%%%%%%%%%%%%%%%%%%%
%% Ваше решение задачи здесь


\end{solution} 

%%%%%%%%%%%%%%%%%%%%%%%%%%%%%%%%%%%%%%%%%%%%%%%%%%%%
% Задача 2
\begin{problem}{13}
Дан тетраэдр $ABCD$. Найти точку $M$ для которой
$\vec{MA}+\vec{MB}+\vec{MC}+\vec{MD}=\vec{0}$.

\end{problem}
\begin{solution}
%%%%%%%%%%%%%%%%%%%%%%%%%%%%%%%%%%%%%%%%%%%%%%%%%%%%
%% Ваше решение задачи здесь


\end{solution} 

%%%%%%%%%%%%%%%%%%%%%%%%%%%%%%%%%%%%%%%%%%%%%%%%%%%%
% Задача 3
\begin{problem}{35}
Даны радиус-векторы $\vec{r_1}$, $\vec{r_2}$, $\vec{r_3}$ вершин треугольника $ABC$. 
Найти радиус-вектор $\vec r$ точки пересечения его медиан.

\end{problem}
\begin{solution}
%%%%%%%%%%%%%%%%%%%%%%%%%%%%%%%%%%%%%%%%%%%%%%%%%%%%
%% Ваше решение задачи здесь


\end{solution} 

%%%%%%%%%%%%%%%%%%%%%%%%%%%%%%%%%%%%%%%%%%%%%%%%%%%%
% Задача 4
\begin{problem}{24}
В трапеции $ABCD$ отношение основания $BC$ к основанию $AD$ равно $\lambda$. Принимая за базис
векторы $\vec{AD}$ и $\vec{AB}$, найти координаты векторов 
$\vec{AB}$, $\vec{BC}$, $\vec{CD}$, $\vec{DA}$,
$\vec{AC}$ и $\vec{BD}$.

\end{problem}
\begin{solution}
%%%%%%%%%%%%%%%%%%%%%%%%%%%%%%%%%%%%%%%%%%%%%%%%%%%%
%% Ваше решение задачи здесь


\end{solution} 

%%%%%%%%%%%%%%%%%%%%%%%%%%%%%%%%%%%%%%%%%%%%%%%%%%%%
% Задача 5
\begin{problem}{17}
Доказать, что сумма векторов, идущих из центра правильного многоугольника к его вершинам, равна нулю.

\end{problem}
\begin{solution}
%%%%%%%%%%%%%%%%%%%%%%%%%%%%%%%%%%%%%%%%%%%%%%%%%%%%
%% Ваше решение задачи здесь


\end{solution} 

%%%%%%%%%%%%%%%%%%%%%%%%%%%%%%%%%%%%%%%%%%%%%%%%%%%%
% Задача 6
\begin{problem}{29}
Показать, что каковы бы ни были три вектора $\vec a$, $\vec b$ и $\vec c$ и три числа $\alpha$, $\beta$, $\gamma$
векторы $\alpha\vec{a}-\beta\vec{b}$, $\gamma\vec{b}-\alpha\vec{c}$,
$\beta\vec{c}-\gamma\vec{a}$ компланарны.

\end{problem}
\begin{solution}
%%%%%%%%%%%%%%%%%%%%%%%%%%%%%%%%%%%%%%%%%%%%%%%%%%%%
%% Ваше решение задачи здесь


\end{solution} 

%%%%%%%%%%%%%%%%%%%%%%%%%%%%%%%%%%%%%%%%%%%%%%%%%%%%
% Задача 7
\begin{problem}{31}
Даны вектора $\vec{a}=\{1,2,3\}$, $\vec{b}=\{2,-2,1\}$, $\vec{c}=\{4,0,3\}$, $\vec{d}=\{16,10,18\}$.
Найти вектор, являющийся проекцией вектора $\vec d$ на плоскость, определяемую векторами
$\vec a$ и $\vec b$ при направлении проектирования, параллельном вектору $\vec c$.

\end{problem}
\begin{solution}
%%%%%%%%%%%%%%%%%%%%%%%%%%%%%%%%%%%%%%%%%%%%%%%%%%%%
%% Ваше решение задачи здесь


\end{solution} 
\end{document}
