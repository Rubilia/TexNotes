\documentclass[a4paper, 12pt]{article}
\usepackage[T2A]{fontenc}% Внутренняя T2A кодировка TeX
\usepackage[utf8]{inputenc}% кодировка файла
\usepackage[russian]{babel}% поддержка переносов в русском языке
\usepackage{comment}% enables the use of multi-line comments (\ifx \fi) 
\usepackage{fullpage}% changes the margin
\usepackage[a4paper, total={7in, 10in}]{geometry}
\usepackage[fleqn]{amsmath}
\usepackage{amssymb,amsthm}  
\usepackage{graphicx}
\usepackage{verbatim}
\usepackage{float}
\usepackage{xcolor}
\usepackage{mdframed}
\usepackage[shortlabels]{enumitem}
\usepackage{indentfirst}
\usepackage{hyperref}
\usepackage{cancel}

\newenvironment{task}[1][0]{\vspace{.5cm} {\textbf{№ #1} \vspace{.5cm}\\ }}{}

\begin{document}
	\begin{task}[1]
$
\begin{pmatrix}
	3 & 2\\
	5 & -2\\
	-1 & 0\\
\end{pmatrix} \cdot
\begin{pmatrix}
	2 & -3 & 5\\
	4 & 0 & 1\\
\end{pmatrix} \cdot
\begin{pmatrix}
	3 & 2\\
	5 & -2\\
	-1 & 0\\
\end{pmatrix}=
\begin{pmatrix}
	14 & -9 & 17\\
	2 & -15 & 23\\
	-2 & 3 & -5\\
\end{pmatrix} \cdot
\begin{pmatrix}
	3 & 2\\
	5 & -2\\
	-1 & 0\\
\end{pmatrix}=
\begin{pmatrix}
	-20 & 46\\
	-92 & 34\\
	14 & -10\\
\end{pmatrix}
$
\end{task}

\begin{task}[1.1]
$
\begin{pmatrix}
	1 & 2\\
	3 & 4\\
\end{pmatrix} \cdot
\begin{pmatrix}
	0 & -1\\
	1 & 2\\
\end{pmatrix} \cdot
\begin{pmatrix}
	1 & 2\\
	3 & 4\\
\end{pmatrix} \cdot
\begin{pmatrix}
	0 & 1\\
	-1 & 2\\
\end{pmatrix}=
\begin{pmatrix}
	2 & 3\\
	4 & 5\\
\end{pmatrix} \cdot
\begin{pmatrix}
1 & 2\\
3 & 4\\
\end{pmatrix} \cdot
\begin{pmatrix}
0 & 1\\
-1 & 2\\
\end{pmatrix}=
\begin{pmatrix}
	11 & 16\\
	19 & 28\\
\end{pmatrix} \cdot
\begin{pmatrix}
	0 & 1\\
	-1 & 2\\
\end{pmatrix}=\\=
\begin{pmatrix}
	-16 & 43\\
	-28 & 75\\
\end{pmatrix}
$\\
\end{task}

\begin{task}[3]
\begin{description}
	\item[Найдем характеристический полином матрицы:]{
$P(\lambda) = det(A-\lambda E) =\vspace{.3cm}\\=
\begin{vmatrix}
	3-\lambda & 1 & 4 & 4\\
	-4 & -2-\lambda & -4 & -4\\
	-1 & 0 & 2-\lambda & -1\\
	0 & -1 & -4 & -1-\lambda\\
\end{vmatrix}
\underset{\begin{subarray}{c}
		\text{$R_1: R_1+(3-\lambda)R_3$}\\
		\text{$R_2: R_2 - 4 R_3$}\\
\end{subarray}}{=}
\begin{vmatrix}
	0 & 1 & \lambda^2-5\lambda+10 & \lambda + 1\\
	0 & -2-\lambda & 4\lambda -12 & 0\\
	-1 & 0 & 2-\lambda & -1\\
	0 & -1 & -4 & -1-\lambda\\
\end{vmatrix}=\vspace{.3cm}\\=
\begin{vmatrix}
	1 & \lambda^2-5\lambda+10 & \lambda+1\\
	-2-\lambda & 4\lambda-12 & 0\\
	1 & 4 & 1+\lambda\\
\end{vmatrix} = (\lambda + 1) \cdot
\begin{vmatrix}
	1 & \lambda^2-5\lambda + 10 & 1\\
	-2-\lambda & 4\lambda-12 & 0\\
	1 & 4 & 1\\
\end{vmatrix} \: \underset{R_1:R_1-R_3}{=} \vspace{.3cm}\\=
(\lambda+1) \cdot 
\begin{vmatrix}
	0 & \lambda^2-5\lambda+6 & 0\\
	-2-\lambda & 4\lambda-12 & 0\\
	1 & 4 & 1\\
\end{vmatrix}=(\lambda+1) \cdot 
\begin{vmatrix}
	0 & \lambda^2-5\lambda+6\\
	-2-\lambda & 4\lambda-12\\
\end{vmatrix}=\\=(\lambda+1)(\lambda+2)(\lambda-2)(\lambda-3)\\
$
}
\item[Найдем собственные числа матрицы:]{
$\newline
P(\lambda) = 0 \leftrightarrow
\Large{
\left[
\begin{array}{lc}
	\begin{subarray}{l}
		\lambda_1=-2\\
		\lambda_2=-1\\
		\lambda_3=2\\
		\lambda_4=3\\
	\end{subarray} &
	\begin{subarray}{c}
		\text{Собственные}\\
		\text{числа матрицы}
	\end{subarray}
\end{array}
\right.}$
}
\item[\text{Найдем собственные вектора матрицы:}]{$\newline$
\begin{enumerate}
	\item[$1.\:\:\lambda=-2$]{$\newline$
$\left(
\begin{array}{cccc|c}
	5 & 1 & 4 & 4 & 0\\
	-4 & 0 & -4 & -4 & 0\\
	-1 & 0 & 4 & -1 & 0\\
	0 & -1 & -4 & 1 & 0\\
\end{array}
\right) \: \underset{
\begin{subarray}{c}
	\text{$R_1:R_1+R_2+R_3$}\\
	\text{$R_2:\frac{R_2-4R_3}{-20}$}\\
	\text{$R_3: -R_3$}\\
\end{subarray}
}{=} \:
\left(
\begin{array}{cccc|c}
	0 & 1 & 4 & -1 & 0\\
	0 & 0 & 1 & 0 & 0\\
	1 & 0 & -4 & 1 & 0\\
	0 & -1 & -4 & 1 & 0\\	
\end{array}
\right)\: \underset{\begin{subarray}{c}
		\text{$R_4:R_4+R_1$}\\
		\text{$R_3:R_3+R_1$}\\
		\text{$R_1: R_1-4R_2$}\\
\end{subarray}}{=} \\=
\left(
\begin{array}{cccc|c}
	1 & 1 & 0 & 0 & 0\\
	0 & 1 & 0 & -1 & 0\\
	0 & 0 & 1 & 0 & 0\\
	0 & 0 & 0 & 0 & 0\\
\end{array}
\right) \leftrightarrow
\begin{cases}
	x_1+x_2=0\\
	x_2-x_4=0\\
	x_3=0\\
	x_4=c_4\\
\end{cases} \leftrightarrow
\begin{cases}
	x_1=-c_4\\
	x_2=c_4\\
	x_3=0\\
	x_4=c_4\\
\end{cases} \leftrightarrow
X = c_4 \cdot
\begin{pmatrix}
	-1\\1\\0\\1\\
\end{pmatrix}
$
}
\item[$2.\:\:\lambda=-1$]$\newline$
\end{enumerate}
}
\end{description}
\end{task}
\end{document}