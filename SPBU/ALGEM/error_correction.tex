\documentclass[a4paper, 12pt]{article}
\usepackage[T2A]{fontenc}% Внутренняя T2A кодировка TeX
\usepackage[utf8]{inputenc}% кодировка файла
\usepackage[russian]{babel}% поддержка переносов в русском языке
\usepackage{comment}% enables the use of multi-line comments (\ifx \fi) 
\usepackage{fullpage}% changes the margin
\usepackage[a4paper, total={7in, 10in}]{geometry}
\usepackage[fleqn]{amsmath}
\usepackage{amssymb,amsthm}  
\usepackage{graphicx}
\usepackage{verbatim}
\usepackage{float}
\usepackage{xcolor}
\usepackage{mdframed}
\usepackage[shortlabels]{enumitem}
\usepackage{indentfirst}
\usepackage{hyperref}
\usepackage{cancel}

\newenvironment{task}[1][0]{\vspace{.5cm} {\textbf{№ #1} \vspace{.5cm}\\ }}{}

\begin{document}
	\begin{task}[1]
$
\begin{pmatrix}
	3 & 2\\
	5 & -2\\
	-1 & 0\\
\end{pmatrix} \cdot
\begin{pmatrix}
	2 & -3 & 5\\
	4 & 0 & 1\\
\end{pmatrix} \cdot
\begin{pmatrix}
	3 & 2\\
	5 & -2\\
	-1 & 0\\
\end{pmatrix}=
\begin{pmatrix}
	14 & -9 & 17\\
	2 & -15 & 23\\
	-2 & 3 & -5\\
\end{pmatrix} \cdot
\begin{pmatrix}
	3 & 2\\
	5 & -2\\
	-1 & 0\\
\end{pmatrix}=
\begin{pmatrix}
	-20 & 46\\
	-92 & 34\\
	14 & -10\\
\end{pmatrix}
$
\end{task}

\begin{task}[1.1]
$
\begin{pmatrix}
	1 & 2\\
	3 & 4\\
\end{pmatrix} \cdot
\begin{pmatrix}
	0 & -1\\
	1 & 2\\
\end{pmatrix} \cdot
\begin{pmatrix}
	1 & 2\\
	3 & 4\\
\end{pmatrix} \cdot
\begin{pmatrix}
	0 & 1\\
	-1 & 2\\
\end{pmatrix}=
\begin{pmatrix}
	2 & 3\\
	4 & 5\\
\end{pmatrix} \cdot
\begin{pmatrix}
1 & 2\\
3 & 4\\
\end{pmatrix} \cdot
\begin{pmatrix}
0 & 1\\
-1 & 2\\
\end{pmatrix}=
\begin{pmatrix}
	11 & 16\\
	19 & 28\\
\end{pmatrix} \cdot
\begin{pmatrix}
	0 & 1\\
	-1 & 2\\
\end{pmatrix}=\\=
\begin{pmatrix}
	-16 & 43\\
	-28 & 75\\
\end{pmatrix}
$\\
\end{task}

\begin{task}[3]
\begin{description}
	\item[Найдем характеристический полином матрицы:]{
$P(\lambda) = det(A-\lambda E) =\vspace{.3cm}\\=
\begin{vmatrix}
	3-\lambda & 1 & 4 & 4\\
	-4 & -2-\lambda & -4 & -4\\
	-1 & 0 & 2-\lambda & -1\\
	0 & -1 & -4 & -1-\lambda\\
\end{vmatrix}
\underset{\begin{subarray}{c}
		\text{$R_1: R_1+(3-\lambda)R_3$}\\
		\text{$R_2: R_2 - 4 R_3$}\\
\end{subarray}}{=}
\begin{vmatrix}
	0 & 1 & \lambda^2-5\lambda+10 & \lambda + 1\\
	0 & -2-\lambda & 4\lambda -12 & 0\\
	-1 & 0 & 2-\lambda & -1\\
	0 & -1 & -4 & -1-\lambda\\
\end{vmatrix}=\vspace{.3cm}\\=
\begin{vmatrix}
	1 & \lambda^2-5\lambda+10 & \lambda+1\\
	-2-\lambda & 4\lambda-12 & 0\\
	1 & 4 & 1+\lambda\\
\end{vmatrix} = (\lambda + 1) \cdot
\begin{vmatrix}
	1 & \lambda^2-5\lambda + 10 & 1\\
	-2-\lambda & 4\lambda-12 & 0\\
	1 & 4 & 1\\
\end{vmatrix} \: \underset{R_1:R_1-R_3}{=} \vspace{.3cm}\\=
(\lambda+1) \cdot 
\begin{vmatrix}
	0 & \lambda^2-5\lambda+6 & 0\\
	-2-\lambda & 4\lambda-12 & 0\\
	1 & 4 & 1\\
\end{vmatrix}=(\lambda+1) \cdot 
\begin{vmatrix}
	0 & \lambda^2-5\lambda+6\\
	-2-\lambda & 4\lambda-12\\
\end{vmatrix}=\\=\underbrace{(\lambda+1)(\lambda+2)(\lambda-2)(\lambda-3)}_{P(\lambda)}\\
$
}
\item[Найдем собственные числа матрицы:]{
$\newline
P(\lambda) = 0 \leftrightarrow
\Large{
\left[
\begin{array}{lc}
	\begin{subarray}{l}
		\lambda_1=-2\\
		\lambda_2=-1\\
		\lambda_3=2\\
		\lambda_4=3\\
	\end{subarray} &
	\begin{subarray}{c}
		\text{Собственные}\\
		\text{числа матрицы}
	\end{subarray}
\end{array}
\right.}$
}
\item[\text{Найдем собственные вектора матрицы:}]{$\newline$
\begin{enumerate}
	\item[$1.\:\:\lambda=-2$]{$\newline$
$\left(
\begin{array}{cccc|c}
	5 & 1 & 4 & 4 & 0\\
	-4 & 0 & -4 & -4 & 0\\
	-1 & 0 & 4 & -1 & 0\\
	0 & -1 & -4 & 1 & 0\\
\end{array}
\right) \: \underset{
\begin{subarray}{c}
	\text{$R_1:R_1+R_2+R_3$}\\
	\text{$R_2:\frac{R_2-4R_3}{-20}$}\\
	\text{$R_3: -R_3$}\\
\end{subarray}
}{=} \:
\left(
\begin{array}{cccc|c}
	0 & 1 & 4 & -1 & 0\\
	0 & 0 & 1 & 0 & 0\\
	1 & 0 & -4 & 1 & 0\\
	0 & -1 & -4 & 1 & 0\\	
\end{array}
\right)\: \underset{\begin{subarray}{c}
		\text{$R_4:R_4+R_1$}\\
		\text{$R_3:R_3+R_1$}\\
		\text{$R_1: R_1-4R_2$}\\
\end{subarray}}{=} \\=
\left(
\begin{array}{cccc|c}
	1 & 1 & 0 & 0 & 0\\
	0 & 1 & 0 & -1 & 0\\
	0 & 0 & 1 & 0 & 0\\
	0 & 0 & 0 & 0 & 0\\
\end{array}
\right) \leftrightarrow
\begin{cases}
	x_1+x_2=0\\
	x_2-x_4=0\\
	x_3=0\\
	x_4=c_4\\
\end{cases} \leftrightarrow
\begin{cases}
	x_1=-c_4\\
	x_2=c_4\\
	x_3=0\\
	x_4=c_4\\
\end{cases} \leftrightarrow
X = c_4 \cdot
\underbrace{
\begin{pmatrix}
	-1\\1\\0\\1\\
\end{pmatrix}}_{\vec{v_1}}
$
}
\item[$2.\:\:\lambda=-1$]{$\newline$
$\left(
\begin{array}{cccc|c}
	4 & 1 & 4 & 4 & 0\\
	-4 & -1 & -4 & -4 & 0\\
	-1 & 0 & 3 & -1 & 0\\
	0 & -1 & -4 & 0 & 0\\
\end{array}
\right) \: \underset{
	\begin{subarray}{c}
		\text{$R_1:R_1+R_2$}\\
		\text{$R_2:R_2-4R_3$}\\
		\text{$R_3: -R_3$}\\
		\text{$R_4: -R_4$}\\
	\end{subarray}}{=} \:
\left(
\begin{array}{cccc|c}
	0 & 0 & 0 & 0 & 0\\
	0 & -1 & -16 & 0 & 0\\
	1 & 0 & -3 & 1 & 0\\
	0 & 1 & 4 & 0 & 0\\
\end{array}\right)  \: \underset{
\begin{subarray}{c}
	\text{$R_2: -\frac{R_2+R_4}{12}$}\\
	\text{$R_3: R_3+3R_2$}\\
	\text{$R_4: R_4-4R_2$}\\
\end{subarray}}{=} \:
\left(
\begin{array}{cccc|c}
	1 & 0 & 0 & 1 & 0\\
	0 & 1 & 0 & 0 & 0\\
	0 & 0 & 1 & 0 & 0\\
	0 & 0 & 0 & 0 & 0\\
\end{array}
\right)\\
\leftrightarrow
\begin{cases}
	x_1+x_4=0\\
	x_2=0\\
	x_3=0\\
	x_4=c_4\\
\end{cases} \leftrightarrow
\begin{cases}
	x_1=-c_4\\
	x_2 = 0\\
	x_3 = 0\\
	x_4 = c_4\\
\end{cases} \leftrightarrow
X = c_4 \cdot \underbrace{
\begin{pmatrix}
	-1\\0\\0\\1\\
\end{pmatrix}}_{\vec{v_2}}
$
}
\item[$3.\:\:\lambda=2$]{$\newline$
$\left(
\begin{array}{cccc|c}
	1 & 1 & 4 & 4 & 0\\
	-4 & -4 & -4 & -4 & 0\\
	-1 & 0 & 0 & -1 & 0\\
	0 & -1 & -4 & -3 & 0\\
\end{array}
\right) \: \underset{\begin{subarray}{c}
	\text{$R_2: -\frac{R_2}{4}$}\\
	\text{$R_3: -R_3$}\\
	\text{$R_4: -R_4$}\\
	\text{$R_1: \frac{R_1-R_2}{3}$}\\
\end{subarray}}{=}\:
\left(
\begin{array}{cccc|c}
	0 & 0 & 1 & 1 & 0\\
	1 & 1 & 1 & 1 & 0\\
	1 & 0 & 0 & 1 & 0\\
	0 & 1 & 4 & 3 & 0\\
\end{array}
\right) \: \underset{
\begin{subarray}{c}
	\text{$R_2:R_2-R_1-R_3$}\\
	\text{$R_4: -(R_4-R_2-4R_1)$}\\
\end{subarray}}{=} \:
\left(
\begin{array}{cccc|c}
	1 & 0 & 0 & 1 & 0\\
	0 & 1 & 0 & -1 & 0\\
	0 & 0 & 1 & 1 & 0\\
	0 & 0 & 0 & 0 & 0\\
\end{array}
\right)\\
\leftrightarrow
\begin{cases}
	x_1 + x_4=0\\
	x_2-x_4=0\\
	x_3+x_4=0\\
	x_4=c_4\\
\end{cases} \leftrightarrow
\begin{cases}
	x_1=-c_4\\
	x_2=c_4\\
	x_3=-c_4\\
	x_4=c_4\\
\end{cases} \leftrightarrow
X = c_4 \cdot \underbrace{
\begin{pmatrix}
	-1\\1\\-1\\1
\end{pmatrix}}_{\vec{v_3}}
$
}
\item[$4.\:\:\lambda=3$]{$\newline$
$\left(
\begin{array}{cccc|c}
	0 & 1 & 4 & 4 & 0\\
	-4 & -5 & -4 & -4 & 0\\
	-1 & 0 & -1 & -1 & 0\\
	0 & -1 & -4 & -4 & 0\\
\end{array}
\right) \: \underset{
\begin{subarray}{c}
	\text{$R_4: R_4+R_1$}\\
	\text{$R_2: \frac{4R_3-R_2}{5}$}\\
	\text{$R_3: -R_3$}\\
\end{subarray}}{=} \:
\left(
\begin{array}{cccc|c}
	0 & 1 & 4 & 4 & 0\\
	0 & 1 & 0 & 0 & 0\\
	1 & 0 & 1 & 1 & 0\\
	0 & 0 & 0 & 0 & 0\\
\end{array}\right) \: \underset{R_1: \frac{R_1-R_2}{4}}{=} \:
\left(
\begin{array}{cccc|c}
	1 & 0 & 1 & 1 & 0\\
	0 & 1 & 0 & 0 & 0\\
	0 & 0 & 1 & 1 & 0\\
	0 & 0 & 0 & 0 & 0\\
\end{array}
\right) \leftrightarrow\\\leftrightarrow
\begin{cases}
	x_1+x_3+x_4 = 0\\
	x_2=0\\
	x_3+x_4=0\\
	x_4=c_4\\
\end{cases} \leftrightarrow
\begin{cases}
	x_1 = 0\\
	x_2 = 0\\
	x_3=-c_4\\
	x_4=c_4\\
\end{cases} \leftrightarrow X = c_4 \cdot \underbrace{
\begin{pmatrix}
	0\\0\\-1\\1\\
\end{pmatrix}}_{\vec{v_4}}\\
$
}
\end{enumerate}
}
\end{description}

\vspace{.5cm}
\textbf{Ответ:} $P(\lambda)=(\lambda+2)(\lambda+1)(\lambda-2)(\lambda-3)$\\
Собственные числа матрицы: $-2, -1, 2, 3$\\
Собственные вектора матрицы: 
$\begin{pmatrix}-1\\1\\0\\1\\\end{pmatrix}, \begin{pmatrix}-1\\0\\0\\1\\\end{pmatrix}, \begin{pmatrix}-1\\1\\-1\\1\\\end{pmatrix},
\begin{pmatrix}0\\0\\-1\\1\\\end{pmatrix}$
\end{task}


\newpage
\begin{task}[3.1]
\begin{description}
\item[Найдем характеристический полином матрицы:]{
	$P(\lambda) = det(A-\lambda E) =\vspace{.3cm}\\=
\begin{vmatrix}
	7-\lambda & -3 & 2 & -6\\
	1 & -1-\lambda & -2 & -6\\
	-1 & 3 & 4-\lambda & 6\\
	-1 & -3 & 2 & 2-\lambda\\
\end{vmatrix} \: \underset{
\begin{subarray}{c}
	\text{$R_1: R_1+R_2(\lambda-7)$}\\
	\text{$R_3: R_3+R_2$}\\
	\text{$R_4:R_4+R_2$}\\
\end{subarray}}{=} \:
\begin{vmatrix}
	0 & -\lambda^2+6\lambda+4 & 16-2\lambda & 36-6\lambda\\
	1 & -1-\lambda & -2 & -6\\
	0 & 2-\lambda & 2-\lambda & 0\\
	0 & -4-\lambda & 0 & -4-\lambda\\
\end{vmatrix}=\\=(\lambda-2)(\lambda+4) \cdot
\begin{vmatrix}
	\lambda^2-6\lambda-4 & 2\lambda - 16 & 6\lambda-36\\
	1 & 1 & 0\\
	1 & 0 & 1\\
\end{vmatrix} = (\lambda-2)(\lambda+4) \cdot [
\begin{vmatrix}
	2\lambda-16 & 6\lambda-36\\
	1 & 0\\
\end{vmatrix} +
\begin{vmatrix}
	\lambda^2-6\lambda-4 & 2\lambda-16\\
	1 & 1\\
\end{vmatrix}
] = (\lambda-2)(\lambda+4) \cdot [36-6\lambda +\lambda^2-6\lambda-4 + 16 -2\lambda]=\\=(\lambda-2)(\lambda+4)(\lambda ^2 -14\lambda + 48)=\underbrace{(\lambda+4)(\lambda-2)(\lambda-6)(\lambda-8)}_{P(\lambda)}
$
}
\item[Найдем собственные числа матрицы:]{
$\newline
P(\lambda) = 0 \leftrightarrow
\Large{
	\left[
	\begin{array}{lc}
		\begin{subarray}{l}
			\lambda_1=-4\\
			\lambda_2=2\\
			\lambda_3=6\\
			\lambda_4=8\\
		\end{subarray} &
		\begin{subarray}{c}
			\text{Собственные}\\
			\text{числа матрицы}
		\end{subarray}
	\end{array}
	\right.}$

}
\item[\text{Найдем собственные вектора матрицы:}]{$\newline$
\begin{enumerate}
	\item[$1.\:\:\lambda=-4$]{$\newline$
$
\left(
\begin{array}{cccc|c}
	11 & -3 & 2 & -6 & 0\\
	1 & 3 & -2 & -6 & 0\\
	-1 & 3 & 8 & 6 & 0\\
	-1 & -3 & 2 & 6 & 0\\
\end{array}\right)  \: \underset{
\begin{subarray}{c}
	\text{$R_1: \frac{R_1+6R_3+5R_4}{60}$}\\
	\text{$R_4: R_4+R_2$}\\
	\text{$R_3: \frac{R_3+R_2}{6}$}\\
\end{subarray}}{=} \:
\left(
\begin{array}{cccc|c}
	0 & 0 & 1 & 1 & 0\\
	1 & 3 & -2 & -6 & 0\\
	0 & 1 & 1 & 0 & 0\\
	0 & 0 & 0 & 0 & 0\\
\end{array}
\right) 
\: \underset{
	\begin{subarray}{c}
		\text{$R_2: R_2-3R_3+6R_1$}\\
\end{subarray}}{=} \\=
\left(
\begin{array}{cccc|c}
	1 & 0 & 1 & 0 & 0\\
	0 & 1 & 1 & 0 & 0\\
	0 & 0 & 1 & 1 & 0\\
	0 & 0 & 0 & 0 & 0\\
\end{array}\right) \leftrightarrow 
\begin{cases}
	x_1+x_3=0\\
	x_2+x_3=0\\
	x_3+x_4=0\\
	x_4=c_4\\
\end{cases} \leftrightarrow
\begin{cases}
	x_1=c_4\\
	x_2=c_4\\
	x_3=-c_4\\
	x_4=c_4\\
\end{cases} \leftrightarrow X = c_4 \cdot \underbrace{\begin{pmatrix}
1\\1\\-1\\1\\
\end{pmatrix}}_{\vec{v_1}}
$
}
	\item[$2.\:\:\lambda=2$]{$\newline$
$
\left(
\begin{array}{cccc|c}
	5 & -3 & 2 & -6 & 0\\
	1 & -3 & -2 & -6 & 0\\
	-1 & 3 & 2 & 6 & 0\\
	-1 & -3 & 2 & 0 & 0\\
\end{array}\right)  \: \underset{
\begin{subarray}{c}
	\text{$R_1: \frac{R_1+3R_3+2R_4}{12}$}\\
	\text{$R_4: \frac{R_3-R_4}{6}$}\\
	\text{$R_3: R_3+R_2$}\\
\end{subarray}}{=} \:
\left(
\begin{array}{cccc|c}
	0 & 0 & 1 & 1 & 0\\
	1 & -3 & -2 & -6 & 0\\
	0 & 0 & 0 & 0 & 0\\
	0 & 1 & 0 & 1 & 0\\
\end{array}
\right) \underset{
\begin{subarray}{c}
	\text{$R_2: R_2+3R_4+3R_1$}\\
\end{subarray}}{=} \\=
\left(
\begin{array}{cccc|c}
	1 & 0 & 1 & 0 & 0\\
	0 & 1 & 0 & 1 & 0\\
	0 & 0 & 1 & 1 & 0\\
	0 & 0 & 0 & 0 & 0\\
\end{array}
\right) \leftrightarrow
\begin{cases}
	x_1+x_3=0\\
	x_2+x_4=0\\
	x_3+x_4=0\\
	x_4=c_4\\
\end{cases} \leftrightarrow
\begin{cases}
	x_1=c_4\\
	x_2=-c_4\\
	x_3=-c_4\\
	x_4=c_4\\
\end{cases} \leftrightarrow X = c_4 \cdot
\underbrace{
\begin{pmatrix}
	1\\-1\\-1\\1
\end{pmatrix}}_{\vec{v_2}}\\
$

}
	\item[$3.\:\:\lambda=6$]{$\newline$
$
\left(
\begin{array}{cccc|c}
	1 & -3 & 2 & -6 & 0\\
	1 & -7 & -2 & -6 & 0\\
	-1 & 3 & -2 & 6 & 0\\
	-1 & -3 & 2 & -4 & 0\\
\end{array}
\right) \: \underset{
\begin{subarray}{c}
	\text{$R_3: R_3+R_1$}\\
	\text{$R_2: \frac{R_1-R_2}{4}$}\\
	\text{$R_4: -\frac{R_4+R_1}{2}$}\\
\end{subarray}}{=}
\left(
\begin{array}{cccc|c}
	1 & -3 & 2 & -6 & 0\\
	0 & 1 & 1 & 0 & 0\\
	0 & 0 & 0 & 0 & 0\\
	0 & 3 & -2 & 5 & 0\\
\end{array}\right) \: \underset{
\begin{subarray}{c}
	\text{$R_1: R_1+R_4$}\\
	\text{$R_4: \frac{3R_2-R_4}{5}$}\\
\end{subarray}
}{=} \\=
\left(
\begin{array}{cccc|c}
	1 & 0 & 0 & -1 & 0\\
	0 & 1 & 1 & 0 & 0\\
	0 & 0 & 1 & -1 & 0\\
	0 & 0 & 0 & 0 & 0\\
\end{array}
\right) \leftrightarrow
\begin{cases}
	x_1-x_4=0\\
	x_2+x_3=0\\
	x_3-x_4=0\\
	x_4=c_4
\end{cases} \leftrightarrow
\begin{cases}
	x_1=c_4\\
	x_2=-c_4\\
	x_3=c_4\\
	x_4=c_4
\end{cases} \leftrightarrow
X = c_4 \cdot
\underbrace{
\begin{pmatrix}
	1\\-1\\1\\1\\
\end{pmatrix}}_{\vec{v_3}}
$
}
	\item[$4.\:\:\lambda=8$]{$\newline$
$
\left(
\begin{array}{cccc|c}
	-1 & -3 & 2 & -6 & 0\\
	1 & -9 & -2 & -6 & 0\\
	-1 & 3 & -4 & 6 & 0\\
	-1 & -3 & 2 & -6 & 0\\
\end{array}
\right) \: \underset{
\begin{subarray}{c}
	\text{$R_2: -\frac{R_2+R_3}{6}$}\\
	\text{$R_3: \frac{R_3 - R_1}{6}$}\\
	\text{$R_4: R_4-R_1$}\\
	\text{$R_1: -R_1$}\\
\end{subarray}
}{=} \:
\left(
\begin{array}{cccc|c}
	1 & 3 & -2 & 6 & 0\\
	0 & 1 & 1 & 0 & 0\\
	0 & 1 & -1 & 2 & 0\\
	0 & 0 & 0 & 0 & 0\\
\end{array}
\right) \: \underset{
\begin{subarray}{c}
	\text{$R_1: R_1-3R_3$}\\
	\text{$R_3: \frac{R_2-R_3}{2}$}\\
\end{subarray}
}{=} \\=
\left(
\begin{array}{cccc|c}
	1 & 0 & 1 & 0 & 0\\
	0 & 1 & 1 & 0 & 0\\
	0 & 0 & 1 & -1 & 0\\
	0 & 0 & 0 & 0 & 0\\
\end{array}
\right) \leftrightarrow
\begin{cases}
	x_1+x_3=0\\
	x_2+x_3=0\\
	x_3-x_4=0\\
	x_4=c_4
\end{cases} \leftrightarrow
\begin{cases}
	x_1=-c_4\\
	x_2=-c_4\\
	x_3=c_4\\
	x_4=c_4\\
\end{cases} \leftrightarrow X = c_4 \cdot
\underbrace{
\begin{pmatrix}
	-1\\-1\\1\\1\\
\end{pmatrix}}_{\vec{v_4}}
$
}
\end{enumerate}
}
\end{description}

\vspace{.5cm}
\textbf{Ответ:} $P(\lambda)=(\lambda+4)(\lambda-2)(\lambda-6)(\lambda-8)$\\
Собственные числа матрицы: $-4, 2, 6, 8$\\
Собственные вектора матрицы: 
$\begin{pmatrix}1\\1\\-1\\1\\\end{pmatrix}, \begin{pmatrix}1\\-1\\-1\\1\\\end{pmatrix}, \begin{pmatrix}1\\-1\\1\\1\\\end{pmatrix},
\begin{pmatrix}-1\\-1\\1\\1\\\end{pmatrix}$
\end{task}

\begin{task}[5]
$
\begin{pmatrix}
	7 & 2 & 2 & 4\\
	2 & 4 & 1 & 4\\
	-2 & -1 & 1 & -3\\
	-2 & 1 & 2 & -2\\
\end{pmatrix}^{-1}=?\vspace{.3cm}\\
\left(
\begin{array}{cccc|cccc}
	7 & 2 & 2 & 4 &    1 & 0 & 0 & 0\\
	2 & 4 & 1 & 4 &    0 & 1 & 0 & 0\\
	-2 & -1 & 1 & -3 & 0 & 0 & 1 & 0\\
	-2 & 1 & 2 & -2  & 0 & 0 & 0 & 1\\
\end{array}
\right) \: \underset{
\begin{subarray}{c}
	\text{$R_1: R_1 + 3R_3$}\\
	\text{$R_2: R_2 + R_3$}\\
	\text{$R_3: R_4-R_3$}\\
	\text{$R_4: R_4+2R_1$}\\
\end{subarray}
}{=} \:
\left(
\begin{array}{cccc|cccc}
	1 & -1 & 5 & -5 & 1 & 0 & 3 & 0\\
	0 & 3 & 2 & 1 & 0 & 1 & 1 & 0\\
	0 & 2 & 1 & 1 & 0 & 0 & -1 & 1\\
	0 & -1 & 12 & -12 & 2 & 0 & 6 & 1\\
\end{array}
\right) \: \underset{
\begin{subarray}{c}
	\text{$R_1: R_1-R_4$}\\
	\text{$R_2: R_2+3R_4$}\\
	\text{$R_3: R_3+2R_4$}\\
	\text{$R_4: -R_4$}\\
\end{subarray}
}{=}\\=
\left(
\begin{array}{cccc|cccc}
	1 & 0 & -7 & 7 & -1 & 0 & -3 & -1\\
	0 & 0 & 38 & -35 & 6 & 1 & 19 & 3\\
	0 & 0 & 25 & -23 & 4 & 0 & 11 & 3\\
	0 & 1 & -12 & 12 & -2 & 0 & -6 & -1\\
\end{array}
\right) \:
$
\end{task}
\end{document}