\documentclass[12pt]{article}
\usepackage{mathtools}
\usepackage[utf8]{inputenc}
\usepackage[russian]{babel}
\usepackage{amsfonts}
\usepackage[thinc]{esdiff}
\usepackage{tikz}
\usepackage{pgfplots}
\pgfplotsset{compat=1.15}
\usepackage{mathrsfs}
\usetikzlibrary{arrows}
\pagestyle{empty}

\DeclarePairedDelimiter\abs{\lvert}{\rvert}%

\newcommand*\circled[1]{\tikz[baseline=(char.base)]{
		\node[shape=circle,draw,inner sep=2pt] (char) {#1};}}

\newenvironment{task}[1][0]{\vspace{.5cm} {\textbf{№ #1} \vspace{.5cm}\\ }}{}

\begin{document}
	{\centering \LARGE{Основы дискретной математики}\\\vspace{.5cm}}
\vspace{.5cm}
{
	{\textbf{№ 13г} \vspace{.5cm}\\}
	\textbf{\large{Доказать $A \subseteq (B \cup C)$ \enspace $\leftrightarrow \enspace (A \cap \overline{B}) \subseteq C$}}\vspace{.1cm} \\
	\textbf{В одну сторону:}
	$\forall x \in A$ => $x \in (B \cup C)$
	\begin{enumerate}
		\item {$x \in A \enspace \& \enspace x \in B$\\
		$x \in A \enspace \& \enspace x \notin B \implies A \cap \overline{B} = \emptyset; \quad \emptyset \in C$
	}
	\item{$x \in A \enspace \& \enspace x \notin B \implies x \in A\cap \overline{B}$ 
		=> $x \in C$\\
	
	}
	\end{enumerate}
	\textbf{В другую сторону:}
	\begin{enumerate}
		\item { $x \in A, x \in B \implies x \in (B \cup C)$
		}
		\item{$x \in A, x \notin B, x\in C \implies x \in (B \cup C)$
	
		}
	\end{enumerate}

}

\vspace{.5cm}
{
	{\textbf{№ 14б} \vspace{.5cm}\\}
	\textbf{\large{Доказать: $A \triangle (B \triangle C) = (A \triangle B) \triangle C$\vspace{.2cm} \\}}
	$A \triangle [(B \setminus C) \cap (C \setminus B)] = A \triangle [(B \cap \overline{C})
	\cup (C \cap \overline{B})]$ = $A \triangle X$\\
	Пусть $X = (B \cap \overline{C}) \cup (C \cap \overline{B})$\\
	$\overline{X} = \overline{(B \cap \overline{C}) \cup (C \cap \overline{B})} = 
	(\overline{B} \cup C ) \cap (\overline{C} \cup B)$\\
	$A \triangle X$ = $(A \setminus X) \cup (X \setminus A)$ = $(A \cup \overline{X}) \cup 
	(X \cup \overline{A})$ = $A \cap (\overline{B} \cup C) \cap (\overline{C} \cup B) \cup\\
	\cup [\overline{A} \cap [(B \cap \overline{C}) \cup (C \cap \overline{B})]]$\\
	$X = (\overline{B} \cap \overline{C}) \cup (C \cap \overline{C}) \cup (B \cap \overline{B})
	\cup (B \cap C)$ = $(\overline{B} \cap \overline{C}) \cup (B \cap C)$\\
	$A \triangle X = (A \cap \overline{B} \cap \overline{C}) \cup (A \cap C \cap B) \cup 
	(\overline{A} \cap B \cap \overline{C}) \cup (\overline{A} \cap C \cap B) = 
	[(A \cap \overline{B} \cap \overline{C}) \cup (\overline{A} \cap B \cap \overline{C})] \cup
	(A \cap C \cap B) \cup (\overline{A} \cap C \cap B)$ = $[\overline{C} \cap ((A \cap \overline{B}) \cup (B \cap \overline{A}))] \cup [C \cap ((A \cap \overline{B}) \cup (B \cap \overline{A}))]$ = $[\overline{C} \cap (A \triangle B)] \cup [C \cap (\overline{A \triangle B})]$ = $[A \triangle B] \triangle C$
	
}

\newpage
{
	{\textbf{Задача с листочка № 4} \vspace{.5cm}\\}
	$U =$ $\{\forall$ студентов 3 курса$\}, \quad \abs{U} = 63$\\
	$A = \{$студенты, слушающие прорицание$\}, \quad \abs{A} = 16$\\
	$B = \{$..., магловедение$\}, \quad \abs{B} = 37$\\
	$\abs{A \cap B} = 5$\\

	$\\\abs{A \cup B} = \abs{A \setminus B} + \abs{B \setminus A} - \abs{A \cap B}$
	
	
	
}
{\hfill \textit{04.10.2021}\\ \vspace{.2cm}\\}
{
Def: $A * B = \overline{A \cap B} = \overline{A} \cup \overline{B}$              

\begin{tikzpicture}
	\filldraw[fill=gray] (-2,-2) rectangle (3,2);
	\scope % A \cap B
	\clip (0,0) circle (1);
	\fill[white] (1,0) circle (1);
	\endscope
	% outline
	\draw (0,0) circle (1)
	(1,0) circle (1);
	\draw (0,0) circle (1) (0,1)  node [text=black,above] {$A$}
	(1,0) circle (1) (1,1)  node [text=black,above] {$B$}
	(-2,-2) rectangle (3,2) node [text=black,above] {$U$};
\end{tikzpicture}\\
Доказать: $(A * B) * (A * B) = A \cap B\\$
$A * A = \overline{A} \cup \overline{A} = \overline{A} \implies = \overline{(A * B)} = 
\overline{\overline{A} \cup \overline{B}} = A \cap B$\\
Доказать: $(A * A) * (B * B) = A \cup B\\$
Аналогично: $(A * A) * (B * B) = \overline{A} * \overline{B} = \overline{\overline{A}} \cup \overline{\overline{B}} = A \cup B$
}

\section{Решение уравнений и систем уравнений}
\vspace{.5cm}
{
	{\textbf{№ 1} \vspace{.5cm}\\}
	\large{ Доказать: $A \subseteq B \leftrightarrow A \cup B = B \leftrightarrow A \setminus B = \emptyset \leftrightarrow A \cap \overline{B} = \emptyset$\\}
	\begin{itemize}
		\item {Докажем, что $\overline{A} \cup B = U \implies A \subseteq B:\\$
			$x \in (\overline{A} \cup B) \leftrightarrow x \in \overline{A} \text{ или } x \in B$\\
			$A \cup \overline{A} = U$\\
			Пусть $x \in A$. Если x $\notin B \implies x \notin \overline{A} \implies x \notin \overline{A} \cup B \leftrightarrow x \notin U$ - противоречие $\implies x \in B$ 
		}
	\end{itemize}

	
}

\vspace{.5cm}
{
	{\textbf{№ 2} \vspace{.5cm}\\}
	\large{ Доказать: $A = B \leftrightarrow A \triangle B = \emptyset$\\}
	\begin{enumerate}
		\item {
		$A \triangle B = (A \setminus B) \cup (B \setminus A) = (A \cap \overline{B}) \cup (B \cap \overline{A})$
		\begin{equation*}
			\begin{cases}
				\forall x \in A \implies x \in B\\
				\forall x \in B \implies x \in A
			\end{cases}
			\leftrightarrow
			\begin{cases}
				A \subseteq B \implies A \cap \overline{B} = \emptyset\\
				B \subseteq A \implies B \cap \overline{A} = \emptyset
			\end{cases}
			\implies (A \cap \overline{B}) \cup (B \cap \overline{A}) = \emptyset
		\end{equation*}
	}
	\item{
		$A \triangle B = \emptyset = (A \setminus B) \cup (B \setminus B) \implies A \cap \overline{B} = \emptyset \implies\\\implies A \subseteq B. \text{ Аналогично: } B \subseteq A$\\
	}
	\end{enumerate}
	

}

\vspace{.5cm}
{
	{\textbf{№ 3} \vspace{.5cm}\\}
	\large{ Решить уравнение: $A \setminus X = B$\\
		Пусть $C = A \setminus X \implies C = B \leftrightarrow C \triangle B = \emptyset$\\
		$(A \setminus X) \triangle B = \emptyset = [(A \setminus X) \setminus B] \cup [B \setminus (A \setminus X)] = [(A \cap	\overline{X}) \cap \overline{B}] \cup [B \cap \overline{(A \cap \overline{X})}] = [A \cap \overline{B} \cap \overline{X}] \cup [B \cap \overline{A}] \cup [B \cap X] = \emptyset \leftrightarrow$\\
		\begin{equation*}
			\leftrightarrow
			\begin{cases}
				B \cap \overline{A} = \emptyset\\
				B \cap X = \emptyset\\
				[A \cap \overline{B}] \cap \overline{X} = \emptyset\\
			\end{cases}
		\leftrightarrow
		\begin{cases}
			B \subseteq A \text{}\\
		    X \subseteq \overline{B}\\
			A \cap \overline{B} \subseteq X
		\end{cases}
		\end{equation*}\\
	$A \cap \overline{B} \subseteq X \subseteq \overline{B} \leftrightarrow X = (A \cap \overline{B}) \cup Z, \quad Z \subseteq \overline{A} \cap \overline{B}$
		

	}
}

\vspace{.5cm}
{
	{\textbf{№ 4} \vspace{.5cm}\\}
	\large{ Решить систему:
		\begin{equation*}
			\begin{cases}
				A \cap X = \emptyset\\
				B \cap \overline{X} = \emptyset
			\end{cases} \leftrightarrow
		\begin{cases}
			X \subseteq \overline{A}\\
			B \subseteq X
		\end{cases} \leftrightarrow
		B \subseteq X \subseteq \overline{A} \implies\\
		\end{equation*}\\
	$\implies X = B \cup Y, \quad Y \subseteq \overline{A} \cap \overline{B}$
		
	}
}

\vspace{.5cm}
{
	{\textbf{№ 27 Л\&В} \vspace{.5cm}\\}
	 Решить систему:
		\begin{equation*}
			\begin{cases}
				A \cap X = B\\
				A \cup X = C
			\end{cases} \leftrightarrow
			\begin{cases}
				(A \cap X) \triangle B = \emptyset\\
				(A \cup X) \triangle C = \emptyset
			\end{cases} \leftrightarrow
			\begin{cases}
				[A \cap X \cap \overline{B}] \cup [B \cap \overline{A}] \cup [B \cap \overline{X}] = \emptyset\\
				[A \cap \overline{C}] \cup [X \cap \overline{C}] \cup [C \cap \overline{A} \overline{X}] = \emptyset
			\end{cases} \leftrightarrow
		\end{equation*}\\
		\begin{equation*}
			\leftrightarrow
			\begin{cases}
				B \cap \overline{A} = \emptyset\\
				A \cap \overline{C} = \emptyset\\
				(A \cap \overline{B} \cap X) \cup (\overline{C} \cap X) = [(A \cap \overline{B}) \cup \overline{C}]^{\Large{\text{=S}}} \cap X = \emptyset\\
				[B \cup (C \cap \overline{A})]^{\Large{\text{=T}}} \cap \overline{X} = \emptyset\\
				
			\end{cases} \leftrightarrow
		\end{equation*}
		\begin{equation*}
		\leftrightarrow
		\begin{cases}
			B \subseteq A \subseteq C\\
			T \subseteq X \subseteq \overline{S}\\
			\overline{S} = \overline{(A \cap \overline{B}) \cup \overline{C}} = \overline{A \cap \overline{B}} \cap C = (\overline{A} \cup B) \cap C =\\= (C \cap \overline{A}) \cup (B \cap C) = (C \cap \overline{A}) \cup B = T\\
		\end{cases}
		\leftrightarrow
		\begin{cases}
			X = B \cup (C \cap \overline{A})\\
			B \subseteq A \subseteq C\\
		\end{cases}
		\end{equation*}\\
}

\vspace{.5cm}
{
	{\textbf{№ 28 Л\&В} \vspace{.5cm}\\}
	\begin{equation*}
		\begin{cases}
			A \setminus	X = B\\
			X \setminus A = C
		\end{cases} \leftrightarrow
		\begin{cases}
			(A \setminus X) \triangle B = \emptyset\\
			(X \setminus A) \triangle C = \emptyset
		\end{cases} \leftrightarrow
		\begin{cases}
			(A \cap \overline{X} \cap \overline{B}) \cup (B \cap \overline{A}) \cup (B \cap X) = \emptyset\\
			(X \cap \overline{A} \cap \overline{C}) \cup (C \cap A) \cup (C \cap \overline{X})  = \emptyset\\
		\end{cases}\leftrightarrow
	\end{equation*}
	\begin{equation*}
		\leftrightarrow
		\begin{cases}
			B \subseteq A \subseteq \overline{C}\\
			X \cap [B \cup (\overline{A} \cap \overline{C})] = \emptyset\\
			\overline{X} \cap [C \cup (A \cap \overline{B})] = \emptyset\\
		\end{cases}
			\leftrightarrow
	\begin{cases}
		B \subseteq A \subseteq \overline{C}\\
		[C \cup (A \cap \overline{B})] \subseteq X \subseteq \overline{[B \cup (\overline{A} \cap \overline{C})]}\\
	\end{cases}
	\end{equation*}
	$\overline{[B \cup (\overline{A} \cap \overline{C})]} = \overline{B} \cap \overline{\overline{A} \cap \overline{C}} = \overline{B} \cap (A \cup C) = C \cup (A \cap \overline{B})$
	}

\vspace{.5cm}
{
	{\textbf{№ 30 Л\&В} \vspace{.5cm}\\}
	\begin{equation*}
		\begin{cases}
			A \setminus	X = B\\
			A \cup X = C
		\end{cases} \leftrightarrow
		\begin{cases}
			(A \setminus X) \triangle B = \emptyset\\
			(A \cup X) \triangle C = \emptyset
		\end{cases} \leftrightarrow
		\begin{cases}
			(A \cap \overline{X} \cap \overline{B}) \cup (B \cap \overline{A}) \cup (B \cap X) = \emptyset\\
			(A \cap \overline{C}) \cup (X \cap \overline{C}) \cup (C \cap \overline{A} \cap \overline{X}) = \emptyset
		\end{cases}\leftrightarrow
	\end{equation*}
	\begin{equation*}
		\leftrightarrow
		\begin{cases}
			B \subseteq A \subseteq C\\
			X \cap (B \cup \overline{C}) = \emptyset\\
			\overline{X} \cap ((C \cap \overline{A}) \cup (A \cap \overline{B})) = \emptyset
		\end{cases} \leftrightarrow
		\begin{cases}
			B \subseteq A \subseteq C\\
			((C \cap \overline{A}) \cup (A \cap \overline{B})) \subseteq X \subseteq \overline{B} \cap C
		\end{cases}  \leftrightarrow
	\end{equation*}
	\begin{equation*}
		\leftrightarrow
		\begin{cases}
			X = \overline{B} \cap C\\
			B \subseteq A \subseteq C\\
		\end{cases}
	\end{equation*}
}
\vspace{1cm}\\
\section{Декартово произведение множеств}

$A = \{a_i | ...i \in (1, n)\}$\\
$B = \{b_j| j \in (1, m)\}$\\
$A \times B = \{(a_i; b_j)|i \in (1,n) \& j \in (1, m)\}$\\
$\{x, y\} \times \{1, 2, 3\} = \{(x, 1), (x, 2), (x, 3), (y, 1), (y, 2), (y, 3)\}$\\
$\abs{A \times B} = \abs{A} \cdot \abs{B}$\\
$A \times A = A^2$\\
\subsection{Свойства декартого произведения}
\begin{enumerate}
	\item $A \times B = B \times A \leftrightarrow A = B$
	\item $A \times B = A \times C \leftrightarrow B=C$
\end{enumerate}

\newpage
\section{Отношения}
{\hfill \textit{25.10.2021}\\}

\begin{task}[8e]
{(1, 2), (2, 1), (2, 3), (3, 4), (4, 1)}\\
\begin{enumerate}
	\item Рефлективность - нет (нет пар вида (x, x))
	\item Симметричность - нет
	\item Антисимметричность - нет (есть (2, 1), (1, 2))
	\item{Транзитивность - Да\\
\begin{tabular}{c|c|c|c|c}
	& 1 & 2 & 3 & 4\\
	\hline
	1 & 1 & 1 & 1 & 1\\
	\hline
	2 & 1 & 1 & 1 & 1\\
	\hline
	3 & 1 & 1 & 1 & 1\\
	\hline
	4 & 1 & 1 & 1 & 1\\
\end{tabular}	
}
\end{enumerate}
\end{task}

\begin{task}[2]
$\Phi$ = \{(a, a), (b, b), (c, c), (a, c), (a, d), (b, d), (c, a), (d, a)\}\\
\begin{enumerate}
\item Рефлективность - нет (нет пары (d, d))
\item Симметричность - нет (нет пары (a, c))  $\rightarrow$ +(a, b); (c, b); (d, b)
\item Антисимметричность - нет (есть (a, d), (d, a))
\item{Транзитивность - Да\\
	\begin{tabular}{c|c|c|c|c}
		& a & b & c & d\\
		\hline
		a & 1 & 1 & 1 & 1\\
		\hline
		b & \circled{1} & 1 & \circled{1} & 1\\
		\hline
		c & 1 & 1 & 1 & \circled{1}\\
		\hline
		d & 1 & 1 & 1 & \circled{1}\\
	\end{tabular}	
}
\end{enumerate}
{
\enspace\:\: a\\
\begin{tabular}{c|c}
	a & a\\
	c & c\\
	d & d\\
\end{tabular} $\rightarrow$ +(c, d); (d, c); (d, d)
}\\

{
 b\\
	\begin{tabular}{c|c}
		b & d\\
	\end{tabular}
}\\

{
c\\
	\begin{tabular}{c|c}
		a & a\\
		c & c\\
		d & d\\
	\end{tabular}
}\\

{
d\\
	\begin{tabular}{c|c}
		a & a\\
		b & c\\
		c & d\\
		d &
	\end{tabular} $\rightarrow$ +(b, a); (b, c)
}\\
\end{task}

\section{Комбинаторика}
\end{document}
