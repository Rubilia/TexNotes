\documentclass[a4paper, 12pt]{article}
\usepackage[T2A]{fontenc}
\usepackage[utf8]{inputenc}
\usepackage[russian]{babel}
\usepackage{amsmath}

\newenvironment{task}[1][0]{\vspace{.5cm} {\textbf{№ #1} \vspace{.5cm}\\ }}{}

\begin{document}
\begin{task}[4]
1.\\
Пусть $\rho=\{(a, a), (b, b), (c, c), (a, b), (a, c), (b, a), (b, c), (c, a), (c, b)\},\\
\sigma = \{(d, d), (e, e), (f, f), (d, f), (d, e), (e, d), (e, f), (f, e), (f, d)\}\\
\rho \Delta \sigma = \{(a, a), (b, b), (c, c), (a, b), (a, c), (b, a), (b, c), (c, a), (c, b), (d, d), (e, e),\\ (f, f), (d, f), (d, e), (e, d), (e, f), (f, e), (f, d)\}$\\
Понятно, что $\rho$ и $\sigma$ транзитивные. Также видно, что $\rho \Delta \sigma$ обладает\\ транзитивностью:\\
$
\begin{array}{c}
a\\
\begin{array}{c|c}
	a & a\\
	b & b\\
	c & c\\
\end{array}
\end{array} \enspace
\begin{array}{c}
	b\\
	\begin{array}{c|c}
		a & a\\
		b & b\\
		c & c\\
	\end{array}
\end{array} \enspace
\begin{array}{c}
	c\\
	\begin{array}{c|c}
		a & a\\
		b & b\\
		c & c\\
	\end{array}
\end{array} \enspace
\begin{array}{c}
	d\\
	\begin{array}{c|c}
		d & d\\
		e & e\\
		f & f\\
	\end{array}
\end{array} \enspace
\begin{array}{c}
	e\\
	\begin{array}{c|c}
		d & d\\
		e & e\\
		f & f\\
	\end{array}
\end{array} \enspace
\begin{array}{c}
	f\\
	\begin{array}{c|c}
		d & d\\
		e & e\\
		f & f\\
	\end{array}
\end{array} \enspace\\\\
$
2. \\
Пусть $\rho=\{(a, a), (b, b), (c, c), (a, b), (a, c), (b, a), (b, c), (c, a), (c, b)\},\\
\sigma = \{(d, d), (e, e), (c, c), (d, c), (d, e), (e, d), (e, c), (c, e), (c, d)\}\\
\varphi = \rho \Delta \sigma = \{(a, a), (b, b), (a, b), (a, c), (b, a), (b, c), (c, a), (c, b), (d, d), (e, e), (d, c), (d, e), (e, d),\\ (e, c), (c, e), (c, d)\}\\$
Заметим, что $c \varphi b, b \varphi c$, но $(c, c) \notin \varphi \implies \varphi$ не обладает транзитивностью\\
\\\\
\emph{Таким образом, $\rho \Delta \sigma$ может как обладать транзитивностью, так и нет.}
\end{task}

\begin{task}[4.1]
$\text{Пусть }\varphi = \rho \cap \sigma\text{, }\{a, b, c\} \subset A$ и $a \varphi b\text{, }b \varphi c \implies
\begin{cases}
	(a, b) \in \rho\\
	(a, b) \in \sigma\\
	(b, c) \in \rho\\
	(b, c) \in \sigma
\end{cases} \underset{\text{$\rho$ и $\sigma$ транзитивны}}{\implies}
\begin{cases}
	(a, c) \in \rho\\
	(a, c) \in \sigma
\end{cases} \implies (a, c) \in \rho \cap \sigma\\\\
$
Таким образом $a \varphi b, b\varphi c \implies a \varphi c \implies \rho \cap \sigma $ транзитвивно\\
\end{task}
\end{document}